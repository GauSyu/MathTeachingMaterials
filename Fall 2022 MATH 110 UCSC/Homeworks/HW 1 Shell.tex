%LaTeX Template 
%%	by	Xu Gao (gausyu@gmail.com)
%%LPPL: http://www.latex-project.org/lppl.txt
%------------------------
%The following is the Preamble
%------------------------
\documentclass[11pt]{article}
%This is the document class. 
%%	Most common: article, beamer, book, etc.
%%	11pt means the default font size is 11pt. One can also use 10pt, 11pt, or 12pt.
%%	See https://en.wikibooks.org/wiki/LaTeX/Document_Structure#Document_classes for explaination.
%------------------------
%Packages
%------------------------
\usepackage[a4paper,total={6in, 9in}]{geometry} 
%This aims to customize page layout
\usepackage[T1]{fontenc}
%Using T1 font
\usepackage{mathtools}
%The main MATH package (for whom may wonder, mathtools load amsmath automatically, so no need \usepackage{amsmath})
\usepackage{amsthm}
%This defines the theorem enviroments
\usepackage{amssymb}
%Math symbols
\usepackage{bm}
%Bold math fonts, provide \bm{} command
\usepackage[scr=rsfs,cal=euler]{mathalpha}
%Math fonts 
%%The package provides means of loading maths alphabets (such as are normally addressed via macros \mathcal, \mathbb, \mathfrak and \mathscr)
%%How to use? "scr=" set what font \mathscr use, "cal=" set what font \mathcal use, "bb=" set what font \mathbb use, and "frak=" set what font \mathfrak use.
\usepackage{tikz}
%Drawing
\usepackage{graphicx}
%Support for graphics
\usepackage{hyperref}
\usepackage[nameinlink]{cleveref}
%Hyper links
\hypersetup{
	colorlinks=true,
	linkcolor=blue,
	urlcolor=magenta,
	citecolor=blue
}
%Color links
\usepackage{etoolbox}
%Programming
\usepackage[shortlabels]{enumitem}
%Enable enumerate modification
\usepackage{tcolorbox}
%provides an environment for coloured and framed text boxes with a heading line.
\tcbset{colback=white}
%------------------------
%Theorem-like Enviroments
%------------------------
\theoremstyle{plain}
%This is the theorem style, plain means blod title and italic body
\newtheorem{theorem}{Theorem}
%This is how you define a theorem-like enviroment.
\newtheorem{lemma}[equation]{Lemma}
\newtheorem{proposition}[equation]{Proposition}
%The option [equation] means the lemmas are numbered following the same system of equation.
\newtheorem*{propstar}{Proposition}
%Star version defines unnumbered enviroments
\theoremstyle{definition}
%This is the theorem style, definition means blod title and normal body
\newtheorem{example}[equation]{Example}
%The option [equation] means the examples are numbered following the same system of equation.
\newtheorem{problem}{Problem}
%This is the enviroment used as problems in HWs.
\theoremstyle{remark}
%This is the theorem style, remark means italic title and normal body
\newtheorem*{remark}{Remark}
\newtheorem*{hint}{Hint}
\newtheorem*{optional}{Optional}
%Remark enviroment
\numberwithin{equation}{problem}
%Let equations be numbered with in each problems.
\NewDocumentEnvironment{solution} {o+b} 
	{\begin{proof}[\IfNoValueTF{#1}{\textbf{Solution}}{\textbf{Solution} (#1)}]#2\end{proof}}{}
%This defines the enviroment for you to write solutions.
%------------------------
\newlist{listinprob}{enumerate}{1}
\setlist[listinprob]{label=(\alph{listinprobi}),
                  ref=\theproblem.(\alph{listinprobi}),
                  noitemsep}
\crefname{listinprobi}{problem}{problems}
\Crefname{listinprobi}{Problem}{Problems}
%This defines a new list type used in the enviroment problem.
%------------------------
%DocumentCommands
%------------------------
%Notations for number sets
\NewDocumentCommand \N {} { \mathbb{N} }%Natural numbers
\NewDocumentCommand \Z {} { \mathbb{Z} }%Integers
\NewDocumentCommand \Q {} { \mathbb{Q} }%Rational Numbers
\NewDocumentCommand \R {} { \mathbb{R} }%Real Numbers
\NewDocumentCommand \C {} { \mathbb{C} }%Complex Numbers
\NewDocumentCommand \F {} { \mathbb{F} }%Field
%Notations for maps
\DeclareMathOperator{\id}{id} % identity
\DeclareMathOperator{\pr}{pr} % projection
\DeclareMathOperator{\pt}{pt} % point
\DeclareMathOperator{\res}{res} % restriction
%PairedDelimiters
\DeclarePairedDelimiterX\abs[1]\lvert\rvert
  { \ifblank{#1}{\:\cdot\:}{#1} }
%abstract value function
\DeclarePairedDelimiterX\norm[1]\lVert\rVert
  { \ifblank{#1}{\:\cdot\:}{#1} }
%the norm function
\DeclarePairedDelimiterX\ceil[1]\lceil\rceil
  { \ifblank{#1}{\:\cdot\:}{#1} }
%ceil function
\DeclarePairedDelimiterX\floor[1]\lfloor\rfloor
  { \ifblank{#1}{\:\cdot\:}{#1} }
%floor function
\DeclarePairedDelimiterX\pairing[2]\langle\rangle
  { \ifblank{#1}{\:\cdot\:}{#1}, \ifblank{#2}{\:\cdot\:}{#2} }
\DeclarePairedDelimiterX\inner[2]\lparen\rparen
  { \ifblank{#1}{\:\cdot\:}{#1}, \ifblank{#2}{\:\cdot\:}{#2} }
%Inner product
\providecommand\given{}
\newcommand\SetSymbol[1][]
  { \nonscript\:#1\vert\allowbreak\nonscript\:\mathopen{} }
\DeclarePairedDelimiterX\Set[1]\{\}
  { \renewcommand\given{\SetSymbol[\delimsize]}#1 }
%a set
%
%Miscellaneous
\NewDocumentCommand \vect { m } { \mathbf{#1} }
%Vector
\RenewDocumentCommand \le {} { \leqslant }
\RenewDocumentCommand \ge {} { \geqslant }
%Change the inequality symbols
\NewDocumentCommand \tforall {} {\quad\text{for all}\quad}
%------------------------
% Additional math symbols, used for the Math 110 course
\DeclareMathOperator*\GCD{GCD}
\DeclareMathOperator*\LCM{LCM}
%------------------------
\allowdisplaybreaks
%Allow a long equation to display in more than one page.

%The following change the default layout of title
\makeatletter
\def\@maketitle{%
	\newpage
	\null
	\vskip 2em%
	\begin{center}%
		\let \footnote \thanks
		\sffamily 
		{\LARGE \@title \par}%
		\vskip 1.5em%
		{\large \@subtitle \par}%
		\vskip 1.5em%
		{\large Student name: \@author \par}%
		\ifdefempty{\@acknowledge}{}%
		{\large With help of: \@acknowledge \par}%
		\vskip 1em%
		{\large \@date}%
	\end{center}%
	\par
	\vskip 1.5em%
}	
\global\let\@subtitle\@empty
\DeclareRobustCommand*{\subtitle}[1]{\gdef\@subtitle{#1}}
\global\let\@acknowledge\@empty
\DeclareRobustCommand*{\acknowledge}[1]{\gdef\@acknowledge{#1}}
\makeatother
%Leave this change alone if you don't know what it means

%------------------------
%Information of the file
%------------------------
\title{Homework 1 (due Oct. 2)}
%The title of the file
\author{[Your Full Name Here]}
%Input your full name here.
% \acknowledge{[Your collaborators]}
%Input your collaborators, if none, comment it.
\subtitle{MATH 110~|~Introduction to Number Theory~|~Fall 2022}
%The subtitle
\date{\today} 
%The date, if commented, the value will be \today
 
%------------------------
%Document starts here
%------------------------
\begin{document}
\maketitle

\begin{quotation}
	Whenever you use a result or claim a statement, provide a \textbf{justification} or a \textbf{proof}, unless it has been covered in the class. In the later case, provide a \textbf{citation} (such as ``by the \emph{2-out-of-3} property of \emph{division}'', ``by Coro. 0.31 in the textbook'', or ``by \cite[Coro. 0.31]{texbook}'').

	You are encouraged to \emph{discuss} the problems with your peers. However, you must write the homework \textbf{by yourself} using your words and \textbf{acknowledge your collaborators}.
	
	\Large Please read the \textbf{HW 1.pdf} first. 
\end{quotation}


\begin{problem}\label{p1}
	This problem is a $3$-varibales analogy of the material covered in class.
	\begin{listinprob}
		\item\label{1.a} (5pts) Prove that there exists no integer solution $(x, y, z)$ to the equation 
		\[
			18x - 27y + 39z = 4.
		\]

%----------------------------------------
\begin{solution} %Do not delete
WRITE YOUR SOLUTION HERE
\end{solution}\clearpage %Do not delete
%----------------------------------------

		\item\label{1.b} (5pts) Find \textbf{an} integer solution $(x, y, z)$ to the equation $18x - 27y + 39z = 6$. 

%----------------------------------------
\begin{solution} %Do not delete
WRITE YOUR SOLUTION HERE
\end{solution}\clearpage %Do not delete
%----------------------------------------
		
		\item[($\ast$c).]\label{1.c} (optional, with extra credit up to 5pts) Find \textbf{all} the integer solutions $(x, y, z)$ to the equation $18x - 27y + 39z = 6$. 
		Your answer should give explicit formulae for $x, y, z$ in terms of two free independent integer parameters $m$ and $n$.
		
	\end{listinprob}
	\begin{remark}
		Can you work out a general algorithm?
	\end{remark}
\end{problem}


%----------------------------------------
\begin{solution} %Do not delete
WRITE YOUR SOLUTION HERE
\end{solution}\clearpage %Do not delete
%----------------------------------------

\begin{problem}\label{p2}
	Let $a, b, c$ be three integers, and let $g=\GCD(a,\GCD(b,c))$. 
	\begin{listinprob}
		\item\label{2.a} (8pts) Prove that $g$ satisfies the following properties: \footnote{
			The properties \ref{2.a.i} and \ref{2.a.ii} together are called the \emph{defining property} or the \emph{universal property} of the notion of the \emph{greatest common divisor of $a$, $b$ and $c$}. Notation: $\GCD(a, b, c)$.
			
			Then \cref{2.a} says that $\GCD(a,\GCD(b, c))$ gives an implementation of $\GCD(a, b, c)$.}
		\begin{enumerate}[(i)]
			\item\label{2.a.i} $g$ is a common divisor of $a$, $b$ and $c$, in other words, we have $g\mid a$, $g\mid b$ and $g\mid c$. 
			\item\label{2.a.ii} If $d$ is any common divisor of $a$, $b$ and $c$, then $d\mid g$.
		\end{enumerate}

%----------------------------------------
\begin{solution} %Do not delete
WRITE YOUR SOLUTION HERE
\end{solution}\clearpage %Do not delete
%----------------------------------------
		
		\item\label{2.b} (2pts) 
		Prove that $g$ is the unique natural number satisfying both \ref{2.a.i} and \ref{2.a.ii}.
	\end{listinprob}

%----------------------------------------
\begin{solution} %Do not delete
WRITE YOUR SOLUTION HERE
\end{solution} %Do not delete
%----------------------------------------

	\begin{optional}[with extra credit up to 2pts]
		During your proof, try to only use the following facts: 1, the \emph{definition} of $\GCD(\:\cdot\:,\:\cdot\:)$, 2, the \emph{transitive} property of $\cdot\mid\cdot$, and 3, the \emph{reflexive} property of $\cdot\mid\cdot$. 
	\end{optional}
	\begin{hint}
		Compare this problem with the fact that $\max\Set{a,b,c} = \max\Set{a,\max\Set{b,c}}$.
	\end{hint}
\end{problem}
\clearpage

\begin{problem}
	Let $a_1,\cdots,a_n$ be $n$ integers. 
	\begin{listinprob}
		\item\label{3.a} (2pts) Mimicking \cref{p2}, give the \emph{defining properties} of the notion of the \emph{greatest common divisor of $a_1,\cdots,a_n$}. Then give an implementation of such a notion in terms of $\GCD(\:\cdot\:,\:\cdot\:)$. 
		
		We will use the notation $\GCD(a_1,\cdots,a_n)$ or $\GCD\limits_{1\le i\le n}a_i$ to denote this notion.

%----------------------------------------
\begin{solution} %Do not delete
WRITE YOUR SOLUTION HERE
\end{solution}\clearpage %Do not delete
%----------------------------------------

		\item\label{3.b} (2pts) Give the \emph{defining properties} of the notion of the \emph{least common multiple of $a_1,\cdots,a_n$}. Then give an implementation of such a notion in terms of $\LCM(\:\cdot\:,\:\cdot\:)$. 
		
		We will use the notation $\LCM(a_1,\cdots,a_n)$ or $\LCM\limits_{1\le i\le n}a_i$ to denote this notion.

%----------------------------------------
\begin{solution} %Do not delete
WRITE YOUR SOLUTION HERE
\end{solution}\clearpage %Do not delete
%----------------------------------------

		\item\label{3.c} (6pts) Mimicking the proof of the attached proposition, show that: 
		\begin{quote}
			For any matrix $(a_{ij})_{1\le i\le n, 1\le j\le m}$ of integers, we have 
			\[
				\LCM_{1\le i\le n}\GCD_{1\le j\le m} a_{ij} \mid 
				\GCD_{1\le j\le m}\LCM_{1\le i\le n} a_{ij}.
			\]
		\end{quote}
		\begin{hint}
			What facts are used in the proof?
		\end{hint}
	\end{listinprob}
	\begin{tcolorbox}
		\begin{propstar}\label{prop}
			Let $(x_{ij})_{1\le i\le n, 1\le j\le m}$ be a matrix of real numbers, then we have 
			\[
				\max_{1\le i\le n}\min_{1\le j\le m} x_{ij} \le 
				\min_{1\le j\le m}\max_{1\le i\le n} x_{ij}.
			\]
		\end{propstar}
		\begin{proof}
			Define $f(i)$ ($1\le i\le n$) to be $\min\limits_{1\le j\le m}x_{ij}$. Then we have 
			\[
				f(i) \le x_{ij} \tforall 1\le i\le n, 1\le j\le m.
			\]
			Therefore, we have 
			\[
				\max_{1\le i\le n}f(i) \le \max_{1\le i\le n} x_{ij} 
				\tforall 1\le j\le m.
			\]
			In particular, we have 
			\[
				\max_{1\le i\le n}f(i) \le \min_{1\le j\le m}\max_{1\le i\le n} x_{ij}
			\]
			as desired.
		\end{proof}
	\end{tcolorbox}
\end{problem}

%----------------------------------------
\begin{solution}[of \cref{3.c}] %Do not delete
WRITE YOUR SOLUTION HERE
\end{solution}\clearpage %Do not delete
%----------------------------------------





%------------------------
\begin{thebibliography}{9}  %Do not delete
%List your references here
\bibitem[TEXT]{texbook}
\emph{An Illustrated Theory of Numbers}, Martin H. Weissman.

\bibitem[label]{citekey}
[Book(s): \emph{Title}, Author ] or [Online: \href{http://example.com/}{Link}]
\end{thebibliography}  %Do not delete
%------------------------
\end{document}
