%LaTeX Template 
%%	by	Xu Gao (gausyu@gmail.com)
%%LPPL: http://www.latex-project.org/lppl.txt
%------------------------
%The following is the Preamble
%------------------------
\documentclass[11pt]{article}
%This is the document class. 
%%	Most common: article, beamer, book, etc.
%%	11pt means the default font size is 11pt. One can also use 10pt, 11pt, or 12pt.
%%	See https://en.wikibooks.org/wiki/LaTeX/Document_Structure#Document_classes for explaination.
%------------------------
%Packages
%------------------------
\usepackage[a4paper,total={6in, 9in}]{geometry} 
%This aims to customize page layout
\usepackage[T1]{fontenc}
%Using T1 font
\usepackage[leqno]{mathtools}
%The main MATH package (for whom may wonder, mathtools load amsmath automatically, so no need \usepackage{amsmath})
\usepackage{amsthm}
%This defines the theorem enviroments
\usepackage{amssymb}
%Math symbols
\usepackage{bm}
%Bold math fonts, provide \bm{} command
\usepackage[scr=rsfs,cal=euler]{mathalpha}
%Math fonts 
%%The package provides means of loading maths alphabets (such as are normally addressed via macros \mathcal, \mathbb, \mathfrak and \mathscr)
%%How to use? "scr=" set what font \mathscr use, "cal=" set what font \mathcal use, "bb=" set what font \mathbb use, and "frak=" set what font \mathfrak use.
\usepackage{tikz}
%Drawing
\usepackage{graphicx}
%Support for graphics
\usepackage{hyperref}
\usepackage[nameinlink]{cleveref}
%Hyper links
\hypersetup{
	colorlinks=true,
	linkcolor=blue,
	urlcolor=magenta
}
%Color links
\usepackage{etoolbox}
%Programming
\usepackage[shortlabels]{enumitem}
%Enable enumerate modification
\usepackage{tcolorbox}
%provides an environment for coloured and framed text boxes with a heading line.
\tcbset{colback=white}
%------------------------
%Theorem-like Enviroments
%------------------------
\crefname{equation}{}{}
\theoremstyle{plain}
%This is the theorem style, plain means blod title and italic body
\newtheorem{theorem}{Theorem}
%This is how you define a theorem-like enviroment.
\newtheorem{lemma}[equation]{Lemma}
\newtheorem{proposition}[equation]{Proposition}
%The option [equation] means the lemmas are numbered following the same system of equation.
\newtheorem*{propstar}{Proposition}
%Star version defines unnumbered enviroments
\theoremstyle{definition}
%This is the theorem style, definition means blod title and normal body
\newtheorem*{defn}{Definition}
\newtheorem{example}[equation]{Example}
%The option [equation] means the examples are numbered following the same system of equation.
\newtheorem{problem}{Problem}
%This is the enviroment used as problems in HWs.
\theoremstyle{remark}
%This is the theorem style, remark means italic title and normal body
\newtheorem*{remark}{Remark}
\newtheorem*{hint}{Hint}
\newtheorem*{optional}{Optional}
%Remark enviroment
\numberwithin{equation}{problem}
%Let equations be numbered with in each problems.
\NewDocumentEnvironment{solution} {o} 
{\begin{proof}[\IfNoValueTF{#1}{\textbf{Solution}}{\textbf{Solution} (#1)}]}{\end{proof}}
%This defines the enviroment for you to write solutions.
%------------------------
\newlist{listinprob}{enumerate}{1}
\setlist[listinprob]{label=(\alph{listinprobi}),
                  ref=\theproblem.(\alph{listinprobi}),
                  noitemsep}
\crefname{listinprobi}{problem}{problems}
\Crefname{listinprobi}{Problem}{Problems}
%This defines a new list type used in the enviroment problem.
%------------------------
%DocumentCommands
%------------------------
%Notations for number sets
\NewDocumentCommand \N {} { \mathbb{N} }%Natural numbers
\NewDocumentCommand \Z {} { \mathbb{Z} }%Integers
\NewDocumentCommand \Q {} { \mathbb{Q} }%Rational Numbers
\NewDocumentCommand \R {} { \mathbb{R} }%Real Numbers
\NewDocumentCommand \C {} { \mathbb{C} }%Complex Numbers
\NewDocumentCommand \F {} { \mathbb{F} }%Field
\NewDocumentCommand \scrO {} {	\mathscr{O}	}%Dedekind doamin
%Notations for maps
\DeclareMathOperator{\id}{id} % identity
\DeclareMathOperator{\pr}{pr} % projection
\DeclareMathOperator{\pt}{pt} % point
\DeclareMathOperator{\res}{res} % restriction
%PairedDelimiters
\DeclarePairedDelimiterX\abs[1]\lvert\rvert
  { \ifblank{#1}{\:\cdot\:}{#1} }
%abstract value function
\DeclarePairedDelimiterX\norm[1]\lVert\rVert
  { \ifblank{#1}{\:\cdot\:}{#1} }
%the norm function
\DeclarePairedDelimiterX\ceil[1]\lceil\rceil
  { \ifblank{#1}{\:\cdot\:}{#1} }
%ceil function
\DeclarePairedDelimiterX\floor[1]\lfloor\rfloor
  { \ifblank{#1}{\:\cdot\:}{#1} }
%floor function
\DeclarePairedDelimiterX\pairing[2]\langle\rangle
  { \ifblank{#1}{\:\cdot\:}{#1}, \ifblank{#2}{\:\cdot\:}{#2} }
\DeclarePairedDelimiterX\inner[2]\lparen\rparen
  { \ifblank{#1}{\:\cdot\:}{#1}, \ifblank{#2}{\:\cdot\:}{#2} }
%Inner product
\providecommand\given{}
\newcommand\SetSymbol[1][]
  { \nonscript\:#1\vert\allowbreak\nonscript\:\mathopen{} }
\DeclarePairedDelimiterX\Set[1]\{\}
  { \renewcommand\given{\SetSymbol[\delimsize]}#1 }
%a set
%
%Miscellaneous
\NewDocumentCommand \vect { m } { \mathbf{#1} }
%Vector
\RenewDocumentCommand \le {} { \leqslant }
\RenewDocumentCommand \ge {} { \geqslant }
%Change the inequality symbols
\NewDocumentCommand \tforall {} {\quad\text{for all}\quad}
%------------------------
% Additional math symbols, used for the Math 110 course
\DeclareMathOperator*\GCD{GCD}
\DeclareMathOperator*\LCM{LCM}
%------------------------
\allowdisplaybreaks
%Allow a long equation to display in more than one page.

%The following change the default layout of title
\makeatletter
\def\@maketitle{%
	\newpage
	\null
	\vskip 2em%
	\begin{center}%
		\let \footnote \thanks
		\sffamily 
		{\LARGE \@title \par}%
		\vskip 1.5em%
		{\large \@subtitle \par}%
		\vskip 1em%
		{\large \@date}%
	\end{center}%
	\par
	\vskip 1.5em%
}	
\global\let\@subtitle\@empty
\DeclareRobustCommand*{\subtitle}[1]{\gdef\@subtitle{#1}}
\makeatother
%Leave this change alone if you don't know what it means

%------------------------
%Information of the file
%------------------------
\title{Homework 2 (due Oct. 10)}
%The title of the file
\author{Xu Gao}
%The author
\subtitle{MATH 110~|~Introduction to Number Theory~|~Fall 2022}
%The subtitle
\date{} 
%The date, if commented, the value will be \today
 
%------------------------
%Document starts here
%------------------------
\begin{document}
\maketitle

\begin{quotation}
	Whenever you use a result or claim a statement, provide a \textbf{justification} or a \textbf{proof}, unless it has been covered in the class. In the later case, provide a \textbf{citation} (such as ``by the \emph{2-out-of-3} property of \emph{division}'' or ``by Coro. 0.31 in the textbook'').

	You are encouraged to \emph{discuss} the problems with your peers. However, you must write the homework \textbf{by yourself} using your words and \textbf{acknowledge your collaborators}.
\end{quotation}

\begin{problem}
	Let $a$, $b$ and $n$ be positive integers. \texttt{Prove} that 
	\begin{listinprob}
		\item (5 pts) $\GCD(a^n,b^n) = \GCD(a,b)^n$ and $\LCM(a^n,b^n) = \LCM(a,b)^n$;
		\item (5 pts) $\GCD(a\cdot n,b\cdot n) = \GCD(a,b)\cdot n$ and $\LCM(a\cdot n,b\cdot n) = \LCM(a,b)\cdot n$;
	\end{listinprob}
\end{problem}

\begin{problem}[10 pts]
	Write the prime factorization of $N = 13!$ and compute $\sigma_{0}(N)$.
	\begin{remark}
		Recall that for any positive integer $n$, we denote by $n!$ (read \textbf{$n$ factorial}) the product of all the integers between $1$ and $n$.
	\end{remark}
\end{problem}

\begin{problem}[10 pts]
	Let $n$ be any positive integer. \texttt{Prove} that there exists a positive integer $k$ (depending on $n$) such that the following list of $n$ consecutive integers:
	\[
		k, k + 1, \cdots, k + n - 1
	\]
	contains \emph{no} prime number at all.
	\begin{hint}
		Use the factorial (but $k=n!$ is NOT the correct answer, start from this and try to see what are missing). You also need the \emph{2-out-of-3} property of division.
	\end{hint}
	\begin{remark}
		From the problem, we can see that the gaps between consecutive prime numbers can be arbitrarily large.
	\end{remark}
\end{problem}

\begin{problem}\label{p4}
	As in class, consider the collection of complex numbers of the form
	\[
		\scrO := \Set*{	a+b\sqrt{-5}	\given	a,b\in\Z	}.
	\]
	\begin{listinprob}
		\item (3 pts) \texttt{Prove} that the set $\scrO$ equipped with the addition and multiplication of complex numbers satisfies the following properties:
		\begin{enumerate}[(i)]
			\item $\scrO$ is closed under addition: for any $\alpha,\beta\in\scrO$, we have $\alpha+\beta\in\scrO$.
			\item $\scrO$ is closed under negation: for any $\alpha\in\scrO$, we have $-\alpha\in\scrO$.
			\item $\scrO$ is closed under multiplication: for any $\alpha,\beta\in\scrO$, we have $\alpha\beta\in\scrO$.
		\end{enumerate}
		\begin{remark}
			In the terms of Algebra, $\scrO$ is a \emph{subring} of the ring $\C$ of complex numbers.
		\end{remark}
		\item\label{Norm} (4 pts) Consider the integer-valued function $N$ defined on $\scrO$: 
		\[
			N(a+b\sqrt{-5}) := a^2+5b^2.
		\]
		\texttt{Prove} that 
		\[
			N(\alpha\beta) = N(\alpha)N(\beta)
		\]
		for any two elements $\alpha$ and $\beta$ in $\scrO$.
		\begin{remark}
			Say that an element $\alpha\in\scrO$ \textbf{divides} another element $\beta\in\scrO$, denoted by $\alpha\mid\beta$ if there is an element $\gamma\in\scrO$ such that $\beta=\alpha\gamma$. Hence, \cref{Norm} shows that 
			\[
				\alpha \mid \beta \implies N(\alpha) \mid N(\beta).
			\]
		\end{remark}
		\item\label{unit} (2 pts) Say that an element $\varepsilon\in\scrO$ is a \textbf{unit} if $\varepsilon$ divides $1$. 
		\texttt{Prove} that all the units in $\scrO$ are $1$ and $-1$.
		\begin{hint}
			Assume $\varepsilon\in\scrO$ is a unit other than $\pm 1$, then use \cref{Norm}.
		\end{hint}
		\item (8 pts) Say that an element $\alpha\in\scrO$ is a \textbf{prime element} if 
		\begin{enumerate}[(i)]
			\item $\alpha$ is nonzero and not a unit;
			\item whenever $\alpha=\gamma\delta$ with $\gamma, \delta \in \scrO$, we necessarily have one of $\gamma, \delta$ being a unit.
		\end{enumerate}
		\texttt{Prove} that the following four elements are prime elements: $2$, $3$, $1+\sqrt{-5}$, and $1-\sqrt{-5}$.
		\begin{hint}
			Proceed by way of contradiction, then use \cref{Norm}.
		\end{hint}
		\item (3 pts) Say that two elements $\alpha,\beta\in\scrO$ are \textbf{associated} if both $\alpha\mid\beta$ and $\beta\mid\alpha$. 
		\texttt{Prove} that none pair of the four elements $2$, $3$, $1+\sqrt{-5}$, and $1-\sqrt{-5}$ are associated. 
		\begin{hint}
			Use the definition of \emph{division} and \cref{unit}.
		\end{hint}
	\end{listinprob}
	\begin{remark}
		A \textbf{prime factorization} of a nonzero element $\alpha\in\scrO$ is a representation
		\[
			\alpha = \varepsilon p_1\cdots p_n,	
		\]
		where $\varepsilon\in\scrO$ is a unit and $p_1,\cdots,p_n\in\scrO$ are prime elements in $\scrO$. 
		Say that $\alpha$ has a \textbf{unique} prime factorization if whenever there is another prime factorization
		\[
			\alpha = \varepsilon' p_1'\cdots p_m', 
		\]
		we necessarily have $m=n$ and there is a bijection $\phi\colon\Set{1,\cdots,n}\to\Set{1,\cdots,m}$ such that each $p_i$ ($1\le i\le n$) is \emph{associated} to $p_{\phi(i)}'$.

		Say that the \textbf{unique prime factorization property} holds in $\scrO$ if any nonzero element $\alpha\in\scrO$ has a \emph{unique prime factorization}. 

		Then \cref{p4} shows that the prime factorization property \textbf{fails} in $\scrO$ due to the following counterexample
		\[
			6 = 2\cdot 3 = (1+\sqrt{-5})\cdot(1-\sqrt{-5}).
		\]
	\end{remark}
\end{problem}

\end{document}
