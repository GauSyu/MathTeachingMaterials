%LaTeX Template 
%%	by	Xu Gao (gausyu@gmail.com)
%%LPPL: http://www.latex-project.org/lppl.txt
%------------------------
%The following is the Preamble
%------------------------
\documentclass[11pt]{article}
%This is the document class. 
%%	Most common: article, beamer, book, etc.
%%	11pt means the default font size is 11pt. One can also use 10pt, 11pt, or 12pt.
%%	See https://en.wikibooks.org/wiki/LaTeX/Document_Structure#Document_classes for explaination.
%------------------------
%Packages
%------------------------
\usepackage[a4paper,total={6in, 9in}]{geometry} 
%This aims to customize page layout
\usepackage[T1]{fontenc}
%Using T1 font
\usepackage[leqno]{mathtools}
%The main MATH package (for whom may wonder, mathtools load amsmath automatically, so no need \usepackage{amsmath})
\usepackage{amsthm}
%This defines the theorem enviroments
\usepackage{amssymb}
%Math symbols
\usepackage{bm}
%Bold math fonts, provide \bm{} command
\usepackage[scr=rsfs,cal=euler]{mathalpha}
%Math fonts 
%%The package provides means of loading maths alphabets (such as are normally addressed via macros \mathcal, \mathbb, \mathfrak and \mathscr)
%%How to use? "scr=" set what font \mathscr use, "cal=" set what font \mathcal use, "bb=" set what font \mathbb use, and "frak=" set what font \mathfrak use.
\usepackage{tikz}
%Drawing
\usepackage{graphicx}
%Support for graphics
\usepackage{hyperref}
\usepackage[nameinlink]{cleveref}
%Hyper links
\hypersetup{
	colorlinks=true,
	linkcolor=blue,
	urlcolor=magenta
}
%Color links
\usepackage{etoolbox}
%Programming
\usepackage[shortlabels]{enumitem}
%Enable enumerate modification
\usepackage{tcolorbox}
%provides an environment for coloured and framed text boxes with a heading line.
\tcbset{colback=white}
%------------------------
%Theorem-like Enviroments
%------------------------
\crefname{equation}{}{}
\theoremstyle{plain}
%This is the theorem style, plain means blod title and italic body
\newtheorem{theorem}{Theorem}
%This is how you define a theorem-like enviroment.
\newtheorem{lemma}[equation]{Lemma}
\newtheorem{proposition}[equation]{Proposition}
%The option [equation] means the lemmas are numbered following the same system of equation.
\newtheorem*{propstar}{Proposition}
%Star version defines unnumbered enviroments
\theoremstyle{definition}
%This is the theorem style, definition means blod title and normal body
\newtheorem*{defn}{Definition}
\newtheorem{example}[equation]{Example}
%The option [equation] means the examples are numbered following the same system of equation.
\newtheorem{problem}{Problem}
%This is the enviroment used as problems in HWs.
\theoremstyle{remark}
%This is the theorem style, remark means italic title and normal body
\newtheorem*{remark}{Remark}
\newtheorem*{hint}{Hint}
\newtheorem*{optional}{Optional}
%Remark enviroment
\numberwithin{equation}{problem}
%Let equations be numbered with in each problems.
\NewDocumentEnvironment{solution} {o} 
{\begin{proof}[\IfNoValueTF{#1}{\textbf{Solution}}{\textbf{Solution} (#1)}]}{\end{proof}}
%This defines the enviroment for you to write solutions.
%------------------------
\newlist{listinprob}{enumerate}{1}
\setlist[listinprob]{label=(\alph{listinprobi}),
                  ref=\theproblem.(\alph{listinprobi})}
\crefname{listinprobi}{problem}{problems}
\Crefname{listinprobi}{Problem}{Problems}
%This defines a new list type used in the enviroment problem.
%------------------------
%DocumentCommands
%------------------------
%Notations for number sets
\NewDocumentCommand \N {} { \mathbb{N} }%Natural numbers
\NewDocumentCommand \Z {} { \mathbb{Z} }%Integers
\NewDocumentCommand \Q {} { \mathbb{Q} }%Rational Numbers
\NewDocumentCommand \R {} { \mathbb{R} }%Real Numbers
\NewDocumentCommand \C {} { \mathbb{C} }%Complex Numbers
\NewDocumentCommand \F {} { \mathbb{F} }%Field
\NewDocumentCommand \scrO {} {	\mathscr{O}	}%Dedekind doamin
%Notations for maps
\DeclareMathOperator{\id}{id} % identity
\DeclareMathOperator{\pr}{pr} % projection
\DeclareMathOperator{\pt}{pt} % point
\DeclareMathOperator{\res}{res} % restriction
%PairedDelimiters
\DeclarePairedDelimiterX\abs[1]\lvert\rvert
  { \ifblank{#1}{\:\cdot\:}{#1} }
%abstract value function
\DeclarePairedDelimiterX\norm[1]\lVert\rVert
  { \ifblank{#1}{\:\cdot\:}{#1} }
%the norm function
\DeclarePairedDelimiterX\ceil[1]\lceil\rceil
  { \ifblank{#1}{\:\cdot\:}{#1} }
%ceil function
\DeclarePairedDelimiterX\floor[1]\lfloor\rfloor
  { \ifblank{#1}{\:\cdot\:}{#1} }
%floor function
\DeclarePairedDelimiterX\pairing[2]\langle\rangle
  { \ifblank{#1}{\:\cdot\:}{#1}, \ifblank{#2}{\:\cdot\:}{#2} }
\DeclarePairedDelimiterX\inner[2]\lparen\rparen
  { \ifblank{#1}{\:\cdot\:}{#1}, \ifblank{#2}{\:\cdot\:}{#2} }
%Inner product
\providecommand\given{}
\newcommand\SetSymbol[1][]
  { \nonscript\:#1\vert\allowbreak\nonscript\:\mathopen{} }
\DeclarePairedDelimiterX\Set[1]\{\}
  { \renewcommand\given{\SetSymbol[\delimsize]}#1 }
%a set
%
%Miscellaneous
\NewDocumentCommand \vect { m } { \mathbf{#1} }
%Vector
\RenewDocumentCommand \le {} { \leqslant }
\RenewDocumentCommand \ge {} { \geqslant }
%Change the inequality symbols
\NewDocumentCommand \tforall {} {\quad\text{for all}\quad}
%------------------------
% Additional math symbols, used for the Math 110 course
\DeclareMathOperator*\GCD{GCD}
\DeclareMathOperator*\LCM{LCM}
\usepackage{derivative}% Support typing calculus notations.
%------------------------
\allowdisplaybreaks
%Allow a long equation to display in more than one page.

%The following change the default layout of title
\makeatletter
\def\@maketitle{%
	\newpage
	\null
	\vskip 2em%
	\begin{center}%
		\let \footnote \thanks
		\sffamily 
		{\LARGE \@title \par}%
		\vskip 1.5em%
		{\large \@subtitle \par}%
		\vskip 1em%
		{\large \@date}%
	\end{center}%
	\par
	\vskip 1.5em%
}	
\global\let\@subtitle\@empty
\DeclareRobustCommand*{\subtitle}[1]{\gdef\@subtitle{#1}}
\makeatother
%Leave this change alone if you don't know what it means

%------------------------
%Information of the file
%------------------------
\title{Homework 3 (due Oct. 20)}
%The title of the file
\author{Xu Gao}
%The author
\subtitle{MATH 110~|~Introduction to Number Theory~|~Fall 2022}
%The subtitle
\date{} 
%The date, if commented, the value will be \today
 
%------------------------
%Document starts here
%------------------------
\begin{document}
\maketitle

\begin{quotation}
	Whenever you use a result or claim a statement, provide a \textbf{justification} or a \textbf{proof}, unless it has been covered in the class. In the later case, provide a \textbf{citation} (such as ``by the \emph{2-out-of-3} property of \emph{division}'' or ``by Coro. 0.31 in the textbook'').

	You are encouraged to \emph{discuss} the problems with your peers. However, you must write the homework \textbf{by yourself} using your words and \textbf{acknowledge your collaborators}.
\end{quotation}

\begin{problem}
	For this problem, you may want to review one-variable Calculus
	\begin{listinprob}
		\item (3 pts) Recall the definition (In this course, $\log=\log_{e}$ denotes the \emph{natural logarithm})
									\[
										\mathrm{Li}(x):=
											\int_{2}^{x}\frac{\odif{t}}{\log{t}}
										\qquad (x>2).
									\]
									\textbf{Question:} What is the  $\odv*{}{x}\mathrm{Li}(x)$ of $\mathrm{Li}(x)$?
		\item (5 pts) Two real functions $f(x)$ and $g(x)$ are \emph{asymptotically equal} if 
		\[
			\lim_{x\to\infty} \frac{f(x)}{g(x)} = 1.
		\]
		\textbf{Prove that:} $\mathrm{Li}(x)$ and $\frac{x}{\log{x}}$ are asymptotically equal.
	\end{listinprob}
\end{problem}

\begin{problem}[5 pts]
	Let $p$ be a prime number and $k,l$ be two natural numbers. 
	\textbf{Show that} 
	\[
		\sum_{i=0}^{k}\sigma_{i}(p^l) = \sum_{i=0}^{l}\sigma_{i}(p^k).
	\]
\end{problem}

\begin{problem}[5 pts]
	Let $n$ be a positive integer and $k$ a natural number. 
	\textbf{Show that} 
	\[
		\sigma_k(n) = \sigma_{-k}(n)n^k.
	\]
	Conclude that $n$ is \emph{perfect} if and only if $\sigma_{-1}(n) = 2$.
\end{problem}

\begin{problem}
	We say that a positive integer $n$ is \textbf{square-free} if $n$ is not divisible by $p^2$ for any prime number $p$. (E.g. $15$ and $37$ are square-free, but $24$ and $49$ are not.) 
	Consider the arithmetic function $\mu$ (named after A.F. M\"obius, popularly known for his strip) as follows:
	\[
		\mu(n) := 
		\begin{dcases*}
			1 & if $n=1$,\\
			0 & if $n$ is NOT sqaure-free,\\
			(-1)^{t} & if $n$ is sqaure-free and has exactly $t$ prime divisors.
		\end{dcases*}
	\]
	\begin{listinprob}
		\item (3 pts) \textbf{Compute} $\mu(n)$ for $n=1,\cdots,15$.
		\item (4 pts) \textbf{Prove that} $\mu$ is \emph{multiplicative}. That is, $\mu(ab)=\mu(a)\mu(b)$ whenever $a,b$ are \emph{coprime}.
		\begin{hint}
			Proceed by cases, taking cue from the definition of $\mu$.
		\end{hint}
	\end{listinprob}
\end{problem}

\begin{problem}
	Let $f(n)$ and $g(n)$ be two arithmetic functions. Define $(f \star g)(n)$ by the formula 
	\[
		(f \star g)(n) := \sum_{d\mid n} f(d)g(\frac{n}{d}),
	\]
	where the summation is taken over the set $\mathscr{D}(n):=\Set*{ d \given d \text{ is a divisor of } n }$. 
	The new function $f \star g$ is called the \textbf{convolution} of $f$ and $g$. The idea originates from Fourier analysis.
	\begin{listinprob}[resume]
		\item (4 pts) Let $\id$ denote the function mapping each positive integer $n$ to itself. \textbf{Compute} the values of $(\id \star \mu)(n)$ for $n=1,\cdots,12$.
		\item (2 pts) Let $\delta_{1}$ be the function defined as follows:
		\[
			\delta_{1}(n) := 
			\begin{dcases*}
				1 & if $n=1$,\\
				0 & if otherwise.
			\end{dcases*}
		\]
		\textbf{Prove that} $\delta_{1}\star f = f\star\delta_{1} = f$ for any arithmetic function $f$.
		(In other words, $\delta_{1}$ is the \emph{identity} for the binary operation $\star$.)
		\item (2 pts) Show that $f\star g = g\star f$ for any arithmetic functions $f$ and $g$. (In other words, the binary operation $\star$ is \emph{commutative}.)
		\begin{hint}
			Show that $d\mapsto \frac{n}{d}$ is a bijection from $\mathscr{D}(n)$ to itself.
		\end{hint}
		\item (6 pts) Show that $(f\star g)\star h = f\star (g\star h)$ for any arithmetic functions $f$, $g$, and $h$. (In other words, the binary operation $\star$ is \emph{associative}.)
		\begin{hint}
			Define $f\star g\star h$ as follows:
			\[
				(f\star g\star h)(n):=\sum_{abc=n}f(a)g(b)h(c),
			\]
			where the summation is taken over the set $\mathscr{D}_{3}(n):=\Set*{ (a,b,c)\in\mathscr{D}(n)^3 \given abc=n }$. 
			Show that each of $(f\star g)\star h$ and $f\star (g\star h)$ is equal to $f\star g\star h$ using a bijective map from its summation index set to $\mathscr{D}_{3}(n)$. 
		\end{hint}
		(At this stage, we see that the set of arithmetic functions equipped with the binary operation $\star$ and the element $\delta_{1}$ forms a \emph{commutative monoid}.)
		\item (6 pts) Suppose $f$ and $g$ are two multiplicative functions. \textbf{Prove that} $f \star g$ is a multiplicative function.
		\begin{hint}
			For any copirme pairs $(m,n)$, use the bijection 
			$\Phi\colon \mathscr{D}(m)\times\mathscr{D}(n) \rightarrow \mathscr{D}(mn)$.
		\end{hint}
		(Hence, the subset of \emph{multiplicative} functions forms a \emph{submonoid}.)
	\end{listinprob}
\end{problem}

\end{document}
