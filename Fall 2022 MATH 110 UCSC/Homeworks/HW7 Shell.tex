%LaTeX Template 
%%	by	Xu Gao (gausyu@gmail.com)
%%LPPL: http://www.latex-project.org/lppl.txt
%------------------------
%The following is the Preamble
%------------------------
\documentclass[11pt]{article}
%This is the document class. 
%%	Most common: article, beamer, book, etc.
%%	11pt means the default font size is 11pt. One can also use 10pt, 11pt, or 12pt.
%%	See https://en.wikibooks.org/wiki/LaTeX/Document_Structure#Document_classes for explaination.
%------------------------
%Packages
%------------------------
\usepackage[a4paper,total={6in, 9in}]{geometry} 
%This aims to customize page layout
\usepackage[T1]{fontenc}
%Using T1 font
\usepackage[leqno]{mathtools}
%The main MATH package (for whom may wonder, mathtools load amsmath automatically, so no need \usepackage{amsmath})
\usepackage{amsthm}
%This defines the theorem enviroments
\usepackage{amssymb}
%Math symbols
\usepackage{bm}
%Bold math fonts, provide \bm{} command
\usepackage[scr=rsfs,cal=euler]{mathalpha}
%Math fonts 
%%The package provides means of loading maths alphabets (such as are normally addressed via macros \mathcal, \mathbb, \mathfrak and \mathscr)
%%How to use? "scr=" set what font \mathscr use, "cal=" set what font \mathcal use, "bb=" set what font \mathbb use, and "frak=" set what font \mathfrak use.
\usepackage{tikz}
%Drawing
\usepackage{graphicx}
%Support for graphics
\usepackage{hyperref}
\usepackage[nameinlink]{cleveref}
%Hyper links
\hypersetup{
	colorlinks=true,
	linkcolor=blue,
	urlcolor=magenta
}
%Color links
\usepackage{etoolbox}
%Programming
\usepackage[shortlabels]{enumitem}
%Enable enumerate modification
\usepackage{tcolorbox}
%provides an environment for coloured and framed text boxes with a heading line.
\tcbset{colback=white}
%------------------------
%Theorem-like Enviroments
%------------------------
\crefname{equation}{}{}
\theoremstyle{plain}
%This is the theorem style, plain means blod title and italic body
\newtheorem{theorem}{Theorem}
%This is how you define a theorem-like enviroment.
\newtheorem{lemma}[equation]{Lemma}
\newtheorem{proposition}[equation]{Proposition}
%The option [equation] means the lemmas are numbered following the same system of equation.
\newtheorem*{propstar}{Proposition}
%Star version defines unnumbered enviroments
\theoremstyle{definition}
%This is the theorem style, definition means blod title and normal body
\newtheorem*{defn}{Definition}
\newtheorem{example}[equation]{Example}
%The option [equation] means the examples are numbered following the same system of equation.
\newtheorem{problem}{Problem}
%This is the enviroment used as problems in HWs.
\theoremstyle{remark}
%This is the theorem style, remark means italic title and normal body
\newtheorem*{remark}{Remark}
\newtheorem*{hint}{Hint}
\newtheorem*{optional}{Optional}
%Remark enviroment
\numberwithin{equation}{problem}
%Let equations be numbered with in each problems.
\NewDocumentEnvironment{solution} {o} 
{\begin{proof}[\IfNoValueTF{#1}{\textbf{Solution}}{\textbf{Solution} (#1)}]}{\end{proof}}
%This defines the enviroment for you to write solutions.
%------------------------
\newlist{listinprob}{enumerate}{1}
\setlist[listinprob]{label=(\alph{listinprobi}),
                  ref=\theproblem.(\alph{listinprobi})}
\crefname{listinprobi}{problem}{problems}
\Crefname{listinprobi}{Problem}{Problems}
%This defines a new list type used in the enviroment problem.
%------------------------
%DocumentCommands
%------------------------
%Notations for number sets
\NewDocumentCommand \N {} { \mathbb{N} }%Natural numbers
\NewDocumentCommand \Z {} { \mathbb{Z} }%Integers
\NewDocumentCommand \Q {} { \mathbb{Q} }%Rational Numbers
\NewDocumentCommand \R {} { \mathbb{R} }%Real Numbers
\NewDocumentCommand \C {} { \mathbb{C} }%Complex Numbers
\NewDocumentCommand \F {} { \mathbb{F} }%Field
\NewDocumentCommand \scrO {} {	\mathscr{O}	}%Dedekind doamin
%Notations for maps
\DeclareMathOperator{\id}{id} % identity
\DeclareMathOperator{\pr}{pr} % projection
\DeclareMathOperator{\pt}{pt} % point
\DeclareMathOperator{\res}{res} % restriction
%PairedDelimiters
\DeclarePairedDelimiterX\abs[1]\lvert\rvert
  { \ifblank{#1}{\:\cdot\:}{#1} }
%abstract value function
\DeclarePairedDelimiterX\norm[1]\lVert\rVert
  { \ifblank{#1}{\:\cdot\:}{#1} }
%the norm function
\DeclarePairedDelimiterX\ceil[1]\lceil\rceil
  { \ifblank{#1}{\:\cdot\:}{#1} }
%ceil function
\DeclarePairedDelimiterX\floor[1]\lfloor\rfloor
  { \ifblank{#1}{\:\cdot\:}{#1} }
%floor function
\DeclarePairedDelimiterX\pairing[2]\langle\rangle
  { \ifblank{#1}{\:\cdot\:}{#1}, \ifblank{#2}{\:\cdot\:}{#2} }
\DeclarePairedDelimiterX\inner[2]\lparen\rparen
  { \ifblank{#1}{\:\cdot\:}{#1}, \ifblank{#2}{\:\cdot\:}{#2} }
%Inner product
\providecommand\given{}
\newcommand\SetSymbol[1][]
  { \nonscript\:#1\vert\allowbreak\nonscript\:\mathopen{} }
\DeclarePairedDelimiterX\Set[1]\{\}
  { \renewcommand\given{\SetSymbol[\delimsize]}#1 }
%a set
%
%Miscellaneous
\NewDocumentCommand \vect { m } { \mathbf{#1} }
%Vector
\RenewDocumentCommand \le {} { \leqslant }
\RenewDocumentCommand \ge {} { \geqslant }
%Change the inequality symbols
\NewDocumentCommand \tforall {} {\quad\text{for all}\quad}
%------------------------
% Additional math symbols, used for the Math 110 course
\DeclareMathOperator*\GCD{GCD}
\DeclareMathOperator*\LCM{LCM}
\usepackage{derivative}% Support typing calculus notations.
%------------------------
\allowdisplaybreaks
%Allow a long equation to display in more than one page.

%The following change the default layout of title
\makeatletter
\def\@maketitle{%
	\newpage
	\null
	\vskip 2em%
	\begin{center}%
		\let \footnote \thanks
		\sffamily 
		{\LARGE \@title \par}%
		\vskip 1.5em%
		{\large \@subtitle \par}%
		\vskip 1.5em%
		{\large Student name: \@author \par}%
		\ifdefempty{\@acknowledge}{}%
		{\large With help of: \@acknowledge \par}%
		\vskip 1em%
		{\large \@date}%
	\end{center}%
	\par
	\vskip 1.5em%
}	
\global\let\@subtitle\@empty
\DeclareRobustCommand*{\subtitle}[1]{\gdef\@subtitle{#1}}
\global\let\@acknowledge\@empty
\DeclareRobustCommand*{\acknowledge}[1]{\gdef\@acknowledge{#1}}
\makeatother
%Leave this change alone if you don't know what it means

%%%
%%%	START WORKS
%%%
%------------------------
%Information of the file
%------------------------
\title{Homework 7 (due Nov. 23)}
%The title of the file
\author{[Your Full Name Here]}
%Input your full name here.
% \acknowledge{[Your collaborators]}
%Input your collaborators, if none, comment it.
\subtitle{MATH 110~|~Introduction to Number Theory~|~Fall 2022}
%The subtitle
\date{\today} 
%The date, if commented, the value will be \today
 
%------------------------
%Document starts here
%------------------------
\begin{document}
\maketitle

\begin{problem}
	In what follows, we fix a prime number $p$.
	For $n$ an integer, recall that $v_p(n)$ is the exponent of $p$ appearing in the prime factorization of $n$. Namely, $p^{v_p(n)}\mid n$, while $p^{v_p(n)+1}\nmid n$. Extend this definition to nonzero fractions as follows:
	\[
		v_p(\frac{n}{m}) := v_p(n) - v_p(m).
	\]
	\begin{listinprob}
		\item (2 pts) Show that, if the two fractions $\frac{n}{m}$ and $\frac{n'}{m'}$ represent the same rational number, then $v_p(\frac{n}{m})=v_p(\frac{n'}{m'})$.
	\end{listinprob}
%----------------------------------------
\begin{solution} %Do not delete
WRITE YOUR SOLUTION HERE
\end{solution}\clearpage %Do not delete
%----------------------------------------

	Hence, we obtain a function $v_p\colon\Q^{\times}\to\Z$. (Recall that $\Q^{\times}$ consists of nonzero rational numbers). The \textbf{$p$-adic norm} of a rational number $x$ is defined to be
	\[
		\abs*{x}_p:=
		\begin{dcases*}
			p^{-v_p(x)} & if $x\neq 0$;\\
			0 & if $x=0$.
		\end{dcases*}
	\]
	For example,
	\[
		\abs*{\frac{24}{25}}_{2} = \frac{1}{8},\qquad 
		\abs*{\frac{24}{25}}_{3} = \frac{1}{3},\qquad 
		\abs*{\frac{24}{25}}_{5} = 25.
	\]	
	\begin{listinprob}[resume]
		\item (3 pts) Prove that $\abs*{-x}_p = \abs*{x}_p$, and $\abs*{xy}_p = \abs*{x}_p\abs*{y}_p$.
%----------------------------------------
\begin{solution} %Do not delete
WRITE YOUR SOLUTION HERE
\end{solution}\clearpage %Do not delete
%----------------------------------------

		\item (5 pts) Prove the \emph{ultrametric triangle inequality}
		\[
			\abs*{x+y}_p \leq \max\Set*{\abs*{x}_p,\abs*{y}_p}.
		\]
	\end{listinprob}
	\begin{remark}
		Note that $\max\Set*{\abs*{x}_p,\abs*{y}_p}\le\abs*{x}_p+\abs*{y}_p$. Hence, the ultrametric triangle inequality implies the usual  triangle inequality. The previous two says that $\abs*{}_p$ can be viewed as analogy of the usual Euclidean norm of vectors, or the absolute value of real numbers.
	\end{remark}
%----------------------------------------
\begin{solution} %Do not delete
WRITE YOUR SOLUTION HERE
\end{solution}\clearpage %Do not delete
%----------------------------------------

	For $z\in\Q$, the \textbf{$p$-adic ball} with center $z$ and radius $r\in\R$ is defined to be
	\[
		B_{\abs*{}_p}(z,r):=\Set*{ x\in\Q \given \abs*{x-z}_p\le r }.
	\]
	\begin{listinprob}[resume]
		\item (5 pts) Prove that the $p$-adic ball $B_{\abs*{}_p}(0,1)$ is closed under addition and multiplication.
	\end{listinprob}
%----------------------------------------
\begin{solution} %Do not delete
WRITE YOUR SOLUTION HERE
\end{solution}\clearpage %Do not delete
%----------------------------------------

	Since, clearly $0,1 \in B_{\abs*{}_p}(0,1)$, we have actually proven that $B_{\abs*{}_p}(0,1)$ is a ring. This ring is called the \textbf{non-complete ring of $p$-adic integers} and is usually denoted by $\Z_{(p)}$.
	\begin{listinprob}[resume]
		\item (2 pts) We can explicitly describe $\Z_{(p)}$. Prove that 
		\[
			\Z_{(p)} = 
			\Set*{
				\frac{a}{b}\in \Q \given a,b \in \Z,\ p\nmid b,\ \GCD(a,b)=1
			}.
		\]
%----------------------------------------
\begin{solution} %Do not delete
WRITE YOUR SOLUTION HERE
\end{solution}\clearpage %Do not delete
%----------------------------------------

		\item\label{p1.p-adic_to_congruence} (3 pts) Let $a$ be an integer and $e$ be a positive integer. Describe the $p$-adic ball $B_{\abs*{}_p}(a,p^{-e})$
		using the language of congruence.
	\end{listinprob}
\end{problem}
%----------------------------------------
\begin{solution} %Do not delete
WRITE YOUR SOLUTION HERE
\end{solution}\clearpage %Do not delete
%----------------------------------------


\begin{problem}
	Let $R$ be a ring. A \textbf{polynomial with coefficients in $R$} is an expression 
	\begin{equation}\label{eq:poly}
		f(T) = a_nT^n + \cdots + a_1T + a_0,
	\end{equation}
	where $a_0,\cdots,a_n\in R$. 
	The set of all polynomials with coefficients in $R$ is denoted $R[T]$. 

	Let $f(T)$ be a polynomial as in \cref{eq:poly}. Its \textbf{derivative} is the polynomial
	\[
		f'(T) := na_nT^{n-1} + \cdots + a_1.
	\]
	Note that this definition is formal, not involving any limit. The \textbf{second derivative} $f''(T)$ of $f(T)$ is the derivative of $f'(T)$. In general, the $k$-th derivative $f^{(k)}(T)$ of $f(T)$ is the derivative of $f^{(k-1)}(T)$.
	\begin{listinprob}
		\item (5 pts) Let $a\in R$. Prove the \emph{Taylor expansion}:
		\[
			f(a+T) = f(a) + f'(a)T + \frac{f''(a)}{2!}T^2 + \cdots + \frac{f^{(n)}(a)}{n!}T^n,
		\]
		where $n$ is the degree of $f(T)$.
%----------------------------------------
\begin{solution} %Do not delete
WRITE YOUR SOLUTION HERE
\end{solution}\clearpage %Do not delete
%----------------------------------------

		\item (5 pts) Let $f(T)$ be a polynomial with coefficients in $\Z$ and $k$ a positive integer. Prove that $\frac{1}{k!}f^{(k)}(T)$ has coefficients in $\Z$. That is to say, every coefficient of $f^{(k)}(T)$ is a multiple of $k!$.
	\end{listinprob}
\end{problem}
%----------------------------------------
\begin{solution} %Do not delete
WRITE YOUR SOLUTION HERE
\end{solution}\clearpage %Do not delete
%----------------------------------------


\begin{problem}
	Let $\phi\colon R\to S$ be a map between rings preserving the operations (sum to sum, product to product, zero to zero, and one to one). Then we have a map
	\[
		\phi_{\ast}\colon R[T]\longrightarrow S[T]
	\]
	mapping a polynomial 
	\[
		f(T) = a_nT^n + \cdots + a_1T + a_0 \in R[T],
	\]
	to a polynomial
	\[
		\phi_{\ast}f(T) = \phi(a_n)T^n + \cdots + \phi(a_1)T + \phi(a_0) \in S[T].
	\]
	If this is the case, we say $f(T)$ \textbf{descends} to $\phi_{\ast}f(T)$, or $f(T)$ is a \textbf{lifting} of $\phi_{\ast}f(T)$.

	Let $f(T)$ be a polynomial with coefficients in $R$. Say $r\in R$ is a \textbf{root} of $f(T)$ in $R$ if $f(r)=0$ in $R$. Say $s\in S$ is a \textbf{root} of $f(T)$ in $S$ (through $\phi$) if $\phi_{\ast}f(s)=0$ in $S$. 
	\begin{listinprob}
		\item (2 pts) Show that, if $r\in R$ is a root of $f(T)$ in $R$, then $\phi(r)$ is a root of $f(T)$ in $S$.
	\end{listinprob}
	If this is the case, we say $r$ is a \textbf{lifting} of the root $\phi(r)$ of $f(T)$ to $R$.
%----------------------------------------
\begin{solution} %Do not delete
WRITE YOUR SOLUTION HERE
\end{solution}\clearpage %Do not delete
%----------------------------------------

	\begin{listinprob}[resume]
		\item\label{p3.non-lifting} (3 pts) Give an example to show that even if $\phi\colon R\to S$ is subjective, NOT all roots in $S$ can have a lifting in $R$. 
		\begin{hint}
			Consider $R=\Z$, $S=\Z/m$ (for your favorite $m$), and $\phi$ the natural quotient map $\Z\to\Z/m$. Then consider a polynomial which have no roots in $\Z$.
		\end{hint}
	\end{listinprob}
\end{problem}
%----------------------------------------
\begin{solution} %Do not delete
WRITE YOUR SOLUTION HERE
\end{solution}\clearpage %Do not delete
%----------------------------------------


\begin{problem}
	In what follows, Let $f(T)$ be a polynomial with coefficients in $\Z$. Then for any positive integer $m$, we can talk about roots of $f(T)$ in $\Z/m$ (through the natural quotient map $\Z\to\Z/m$). In particular, we consider $m=p^e$, where $p$ is a prime number and $e$ is a positive integer.
	\begin{listinprob}
		\item (4 pts) Show that, for any $a\in\Z$, we have 
		\[
			f(a+p^eT) \equiv f(a) + f'(a)p^eT \pmod{p^{2e}}.
		\]
		(The congruence relation reads as saying both sides (as polynomials of $T$) descend to the same polynomial with coefficients in $\Z/p^{2e}$.) Note that this is a statement about polynomials not about integers. 
		\begin{remark}
			This implies that $f(a+p^et) \equiv f(a) + f'(a)p^et \pmod{p^{2e}}$ for all $t\in\Z$.
		\end{remark}
%----------------------------------------
\begin{solution} %Do not delete
WRITE YOUR SOLUTION HERE
\end{solution}\clearpage %Do not delete
%----------------------------------------

		\item (5 pts) Finish proving the \emph{Hensel's lemma}: if $\alpha$ is a root of $f(T)$ in $\Z/p^{e}$ and is NOT a root of $f'(T)$ in $\Z/p$, then there is a unique congruence class $\tilde{\alpha}\in\Z/p^{e+e'}$ (where $e'<e$) such that $\tilde{\alpha}$ is a lifting of the root $\alpha\in\Z/p^{e}$ of $f(T)$ to $\tilde{\alpha}\in\Z/p^{e+e'}$. 
		\begin{hint}
			Read the lecture note. You can use the theorem on lifting multiplicative inverse.
		\end{hint}
	\end{listinprob}	
%----------------------------------------
\begin{solution} %Do not delete
WRITE YOUR SOLUTION HERE
\end{solution}\clearpage %Do not delete
%----------------------------------------

	In what follows, we fix a prime number $p$. 
	Say a sequence $(x_n)_{n\in\N}$ of rational numbers is a \textbf{Cauchy sequence with respect to the $p$-adic norm} (a \textbf{Cauchy sequence} for short) if for every positive real number $\varepsilon>0$, there is a positive integer $N$ such that for all natural numbers $m,n>N$,  
	\[
		\abs*{x_m-x_n}_p<\varepsilon.
	\]
	Say a rational number $x\in\Q$ is the \textbf{limit} of a sequence $(x_n)_{n\in\N}$ of rational numbers \textbf{with respect to the $p$-adic norm} if for every positive real number $\varepsilon>0$, there is a positive integer $N$ such that for all natural numbers $n>N$,  
	\[
		\abs*{x_n-x}_p<\varepsilon.
	\]
	Say two Cauchy sequences $(x_n)_{n\in\N}$ and $(y_n)_{n\in\N}$ are \textbf{equivalent} if the sequence $(x_n-y_n)_{n\in\N}$ has the limit $0$.
	\begin{listinprob}[resume]
		\item (3 pts) Prove that, if a sequence $(x_n)_{n\in\N}$ of rational numbers has a limit $x\in\Q$ with respect to the $p$-adic norm, then it is a Cauchy sequence.
%----------------------------------------
\begin{solution} %Do not delete
WRITE YOUR SOLUTION HERE
\end{solution}\clearpage %Do not delete
%----------------------------------------

		\item (5 pts) Finish proving the following version of \emph{Hensel's lemma}: if $x_0$ is an integer such that $p\mid f(x_0)$ but $p\nmid f'(x_0)$, then it can be extended into a unique (up to equivalence) Cauchy sequence $(x_n)_{n\in\N}$ such that the sequence $(f(x_n))_{n\in\N}$ has the limit $0$ with respect to the $p$-adic norm. 
		\begin{hint}
			Using \cref{p1.p-adic_to_congruence} to translate the statement in the language of congruence.
		\end{hint}
%----------------------------------------
\begin{solution} %Do not delete
WRITE YOUR SOLUTION HERE
\end{solution}\clearpage %Do not delete
%----------------------------------------

		\item (3 pts) Give an example to show that NOT every Cauchy sequence has a limit in $\Q$ with respect to the $p$-adic norm.
		\begin{hint}
			You may want to use \cref{p3.non-lifting}. Consider a sequence obtained from the Hensel's limit. 
		\end{hint}
	\end{listinprob}
\end{problem}
%----------------------------------------
\begin{solution} %Do not delete
WRITE YOUR SOLUTION HERE
\end{solution}\clearpage %Do not delete
%----------------------------------------

\begin{thebibliography}{9}  %Do not delete
%List your references here
\bibitem[TEXT]{texbook}
\emph{An Illustrated Theory of Numbers}, Martin H. Weissman.

\bibitem[label]{citekey}
[Book(s): \emph{Title}, Author ] or [Online: \href{http://example.com/}{Link}]
\end{thebibliography}  %Do not delete
%------------------------
\end{document}
