%LaTeX Template 
%%	by	Xu Gao (gausyu@gmail.com)
%%LPPL: http://www.latex-project.org/lppl.txt
%------------------------
%The following is the Preamble
%------------------------
\documentclass[11pt]{article}
%This is the document class. 
%%	Most common: article, beamer, book, etc.
%%	11pt means the default font size is 11pt. One can also use 10pt, 11pt, or 12pt.
%%	See https://en.wikibooks.org/wiki/LaTeX/Document_Structure#Document_classes for explaination.
%------------------------
%Packages
%------------------------
\usepackage[a4paper,total={6in, 9in}]{geometry} 
%This aims to customize page layout
\usepackage[T1]{fontenc}
%Using T1 font
\usepackage[leqno]{mathtools}
%The main MATH package (for whom may wonder, mathtools load amsmath automatically, so no need \usepackage{amsmath})
\usepackage{amsthm}
%This defines the theorem enviroments
\usepackage{amssymb}
%Math symbols
\usepackage{bm}
%Bold math fonts, provide \bm{} command
\usepackage[scr=rsfs,cal=euler]{mathalpha}
%Math fonts 
%%The package provides means of loading maths alphabets (such as are normally addressed via macros \mathcal, \mathbb, \mathfrak and \mathscr)
%%How to use? "scr=" set what font \mathscr use, "cal=" set what font \mathcal use, "bb=" set what font \mathbb use, and "frak=" set what font \mathfrak use.
\usepackage{tikz}
%Drawing
\usepackage{graphicx}
%Support for graphics
\usepackage{hyperref}
\usepackage[nameinlink]{cleveref}
%Hyper links
\hypersetup{
	colorlinks=true,
	linkcolor=blue,
	urlcolor=magenta
}
%Color links
\usepackage{etoolbox}
%Programming
\usepackage[shortlabels]{enumitem}
%Enable enumerate modification
\usepackage{tcolorbox}
%provides an environment for coloured and framed text boxes with a heading line.
\tcbset{colback=white}
%------------------------
%Theorem-like Enviroments
%------------------------
\crefname{equation}{}{}
\theoremstyle{plain}
%This is the theorem style, plain means blod title and italic body
\newtheorem{theorem}{Theorem}
%This is how you define a theorem-like enviroment.
\newtheorem{lemma}[equation]{Lemma}
\newtheorem{proposition}[equation]{Proposition}
%The option [equation] means the lemmas are numbered following the same system of equation.
\newtheorem*{propstar}{Proposition}
%Star version defines unnumbered enviroments
\theoremstyle{definition}
%This is the theorem style, definition means blod title and normal body
\newtheorem*{defn}{Definition}
\newtheorem{example}[equation]{Example}
%The option [equation] means the examples are numbered following the same system of equation.
\newtheorem{problem}{Problem}
%This is the enviroment used as problems in HWs.
\theoremstyle{remark}
%This is the theorem style, remark means italic title and normal body
\newtheorem*{remark}{Remark}
\newtheorem*{hint}{Hint}
\newtheorem*{optional}{Optional}
%Remark enviroment
\numberwithin{equation}{problem}
%Let equations be numbered with in each problems.
\NewDocumentEnvironment{solution} {o} 
{\begin{proof}[\IfNoValueTF{#1}{\textbf{Solution}}{\textbf{Solution} (#1)}]}{\end{proof}}
%This defines the enviroment for you to write solutions.
%------------------------
\newlist{listinprob}{enumerate}{1}
\setlist[listinprob]{label=(\alph{listinprobi}),
                  ref=\theproblem.(\alph{listinprobi})}
\crefname{listinprobi}{problem}{problems}
\Crefname{listinprobi}{Problem}{Problems}
%This defines a new list type used in the enviroment problem.
%------------------------
%DocumentCommands
%------------------------
%Notations for number sets
\NewDocumentCommand \N {} { \mathbb{N} }%Natural numbers
\NewDocumentCommand \Z {} { \mathbb{Z} }%Integers
\NewDocumentCommand \Q {} { \mathbb{Q} }%Rational Numbers
\NewDocumentCommand \R {} { \mathbb{R} }%Real Numbers
\NewDocumentCommand \C {} { \mathbb{C} }%Complex Numbers
\NewDocumentCommand \F {} { \mathbb{F} }%Field
\NewDocumentCommand \scrO {} {	\mathscr{O}	}%Dedekind doamin
%Notations for maps
\DeclareMathOperator{\id}{id} % identity
\DeclareMathOperator{\pr}{pr} % projection
\DeclareMathOperator{\pt}{pt} % point
\DeclareMathOperator{\res}{res} % restriction
%PairedDelimiters
\DeclarePairedDelimiterX\abs[1]\lvert\rvert
  { \ifblank{#1}{\:\cdot\:}{#1} }
%abstract value function
\DeclarePairedDelimiterX\norm[1]\lVert\rVert
  { \ifblank{#1}{\:\cdot\:}{#1} }
%the norm function
\DeclarePairedDelimiterX\ceil[1]\lceil\rceil
  { \ifblank{#1}{\:\cdot\:}{#1} }
%ceil function
\DeclarePairedDelimiterX\floor[1]\lfloor\rfloor
  { \ifblank{#1}{\:\cdot\:}{#1} }
%floor function
\DeclarePairedDelimiterX\pairing[2]\langle\rangle
  { \ifblank{#1}{\:\cdot\:}{#1}, \ifblank{#2}{\:\cdot\:}{#2} }
\DeclarePairedDelimiterX\inner[2]\lparen\rparen
  { \ifblank{#1}{\:\cdot\:}{#1}, \ifblank{#2}{\:\cdot\:}{#2} }
%Inner product
\providecommand\given{}
\newcommand\SetSymbol[1][]
  { \nonscript\:#1\vert\allowbreak\nonscript\:\mathopen{} }
\DeclarePairedDelimiterX\Set[1]\{\}
  { \renewcommand\given{\SetSymbol[\delimsize]}#1 }
%a set
%
%Miscellaneous
\NewDocumentCommand \vect { m } { \mathbf{#1} }
%Vector
\RenewDocumentCommand \le {} { \leqslant }
\RenewDocumentCommand \ge {} { \geqslant }
%Change the inequality symbols
\NewDocumentCommand \tforall {} {\quad\text{for all}\quad}
%------------------------
% Additional math symbols, used for the Math 110 course
\DeclareMathOperator*\GCD{GCD}
\DeclareMathOperator*\LCM{LCM}
\usepackage{derivative}% Support typing calculus notations.
\newcommand{\genlegendre}[4]{%
  \genfrac{(}{)}{}{#1}{#3}{#4}%
  \if\relax\detokenize{#2}\relax\else_{\!#2}\fi
}
\newcommand{\legendre}[3][]{\genlegendre{}{#1}{#2}{#3}}
\newcommand{\dlegendre}[3][]{\genlegendre{0}{#1}{#2}{#3}}
\newcommand{\tlegendre}[3][]{\genlegendre{1}{#1}{#2}{#3}}
%------------------------
\allowdisplaybreaks
%Allow a long equation to display in more than one page.

%The following change the default layout of title
\makeatletter
\def\@maketitle{%
	\newpage
	\null
	\vskip 2em%
	\begin{center}%
		\let \footnote \thanks
		\sffamily 
		{\LARGE \@title \par}%
		\vskip 1.5em%
		{\large \@subtitle \par}%
		\vskip 1.5em%
		{\large Student name: \@author \par}%
		\ifdefempty{\@acknowledge}{}%
		{\large With help of: \@acknowledge \par}%
		\vskip 1em%
		{\large \@date}%
	\end{center}%
	\par
	\vskip 1.5em%
}	
\global\let\@subtitle\@empty
\DeclareRobustCommand*{\subtitle}[1]{\gdef\@subtitle{#1}}
\global\let\@acknowledge\@empty
\DeclareRobustCommand*{\acknowledge}[1]{\gdef\@acknowledge{#1}}
\makeatother
%Leave this change alone if you don't know what it means

%%%
%%%	START WORKS
%%%
%------------------------
%Information of the file
%------------------------
\title{Homework 8 (due Dec. 2)}
%The title of the file
\author{[Your Full Name Here]}
%Input your full name here.
% \acknowledge{[Your collaborators]}
%Input your collaborators, if none, comment it.
\subtitle{MATH 110~|~Introduction to Number Theory~|~Fall 2022}
%The subtitle
\date{\today} 
%The date, if commented, the value will be \today
 
%------------------------
%Document starts here
%------------------------
\begin{document}
\maketitle

\begin{problem}
	Let $p$ be an odd prime. 
	Recall that a primitive root modulo $p$ is an integer $g$ such that $p-1$ is the smallest positive integer $e$ such that
	\[
		g^{e} \equiv 1 \pmod{p}.
	\] 
	\begin{listinprob}
	\item (5 pts) Consider $\F_p^\times = \F_p \setminus \Set*{\overline{0}}$. Show that there is an \emph{isomorphism} (a bijective map preserving addition, multiplication, zero, and one) from $\F_p^\times$ to $\Z/(p-1)$.
	\begin{hint}
		First show that $\F_p^\times = \Set*{ g^e \given 0\le e< p-1 }$, where $g$ is a primitive root. (Why there is a primitive root?)
	\end{hint}
%----------------------------------------
\begin{solution} %Do not delete
WRITE YOUR SOLUTION HERE
\end{solution}\clearpage %Do not delete
%----------------------------------------

	
	\item (5 pts) Use a primitive root $g$ to demonstrate that $-1$ is a quadratic residue modulo $p$ if and only if $p \equiv 1 \pmod{4}$.
%----------------------------------------
\begin{solution} %Do not delete
WRITE YOUR SOLUTION HERE
\end{solution}\clearpage %Do not delete
%----------------------------------------

	
	\item (5 pts) Use a primitive root $g$ to prove the \emph{Wilson Theorem}: $(p-1)!\equiv -1\pmod{p}$.
	\begin{hint}
		First show that $(p-1)! \equiv g^{1+2+\cdots+(p-2)}\pmod{p}$.
	\end{hint}
%----------------------------------------
\begin{solution} %Do not delete
WRITE YOUR SOLUTION HERE
\end{solution}\clearpage %Do not delete
%----------------------------------------


	\item (5 pts) Given a primitive root $g$, and an integer $a\in\Phi(p)$, prove that $a$ is a quadratic residue modulo $p$ if and only if $a \equiv g^e \pmod{p}$ for an even number $e$. Use this to prove the \emph{Euler's Theorem on quadratic residues}:
	\[
		\text{$a$ is a quadratic residue} \iff a^{\frac{p-1}{2}}\equiv 1\pmod{p}.
	\]
	\end{listinprob} 
\end{problem}
%----------------------------------------
\begin{solution} %Do not delete
WRITE YOUR SOLUTION HERE
\end{solution}\clearpage %Do not delete
%----------------------------------------



\begin{problem}[10 pts]
	Let $p$ be an odd prime. Compute the Legendre symbols 
	\[
		\dlegendre{\frac{p-1}{2}}{p}\qquad\text{and}\qquad\dlegendre{\frac{p+3}{2}}{p}.
	\]
	The results should be stated in language of congruence class of $p$ modulo a certain modulus independent of $p$. Namely, the conditions in the results should be of the form:
	\[
		p\equiv \rule{1cm}{0.15mm} \pmod{m},
	\]
	where $m$ is a modulus independent of $p$. 
	\begin{hint}
		Use the complete multiplicativity of Legendre symbol. 
	\end{hint}
\end{problem}
%----------------------------------------
\begin{solution} %Do not delete
WRITE YOUR SOLUTION HERE
\end{solution}\clearpage %Do not delete
%----------------------------------------



\begin{problem}
	Consider the polynomial $f(T)=T^2+T+1$. 
	The purpose of this problem is to figure out for which prime $p$, $f(T)$ is irreducible modulo $p$.
	\begin{listinprob}
	\item (3 pts) Show that $f(T)$ is irreducible modulo $2$.
	\begin{hint}
		Use Problem 2 (a) from HW 6.
	\end{hint}	
	\end{listinprob}
%----------------------------------------
\begin{solution} %Do not delete
WRITE YOUR SOLUTION HERE
\end{solution}\clearpage %Do not delete
%----------------------------------------

	
	Hence, we may assume $p$ is odd. In what follows, we keep this assumption.
	\begin{listinprob}[resume]
	\item\label{P3:complet_square} (3 pts) Find an integer polynomial of the form $(T+a)^2+q$ such that 
	\[
		f(T) \equiv (T+a)^2+q \pmod{p}.
	\]
	\begin{hint}
		Note that $p$ is odd.
	\end{hint}	
%----------------------------------------
\begin{solution} %Do not delete
WRITE YOUR SOLUTION HERE
\end{solution}\clearpage %Do not delete
%----------------------------------------

	
	\item (3 pts) Argue that $f(T)$ is irreducible if and only if $q$ (the leftover term in \ref{P3:complet_square}) is a quadratic non-residue modulo $p$.
	\end{listinprob}
	Equivalently, $f(T)$ is irreducible if and only if 
	\[
		\dlegendre{q}{p} = -1.
	\]
%----------------------------------------
\begin{solution} %Do not delete
WRITE YOUR SOLUTION HERE
\end{solution}\clearpage %Do not delete
%----------------------------------------


	\begin{listinprob}[resume]
	\item (6 pts) Conclude the condition for $f(T)$ being irreducible modulo $p$ in language of congruence of $p$ modulo a certain modulus independent of $p$. Namely, the condition should be of the form:
	\[
		p\equiv \rule{1cm}{0.15mm} \pmod{m},
	\]
	where $m$ is a modulus independent of $p$. 
	\end{listinprob} 
\end{problem}
%----------------------------------------
\begin{solution} %Do not delete
WRITE YOUR SOLUTION HERE
\end{solution}\clearpage %Do not delete
%----------------------------------------

\begin{thebibliography}{9}  %Do not delete
%List your references here
\bibitem[TEXT]{texbook}
\emph{An Illustrated Theory of Numbers}, Martin H. Weissman.

\bibitem[label]{citekey}
[Book(s): \emph{Title}, Author ] or [Online: \href{http://example.com/}{Link}]
\end{thebibliography}  %Do not delete
%------------------------
\end{document}
