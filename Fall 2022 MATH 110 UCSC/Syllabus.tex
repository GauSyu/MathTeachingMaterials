% LTeX: enabled=false
\documentclass[12pt]{article}
\usepackage[a4paper,total={6in, 8.3in},headheight=15pt]{geometry}
\usepackage{fancyhdr}
\usepackage[T1]{fontenc}
\usepackage{titlesec}
\usepackage{hyperref}
\usepackage[dvipsnames,table]{xcolor}
\usepackage{enumitem}
\usepackage{array}
\usepackage{multirow}
% \usepackage{amsmath}
\usepackage{amssymb,mathtools}

\newcolumntype{C}[1]{>{\centering\let\newline\\\arraybackslash\hspace{0pt}}m{#1}}

\def\suptitle{}
\def\subtitle{}
\def\coursenumber{}
\def\coursetitle{}
\def\coursequarter{}
\def\coursedate{}
\NewDocumentCommand \email { m } {\href{mailto:#1}{#1}}

\hypersetup{
	colorlinks=true,
	urlcolor=MidnightBlue
}
\renewcommand\em{\bfseries}

% \setlength\parindent{0pt}

\arrayrulecolor{SkyBlue}

\setlist{nosep} % or \setlist{noitemsep} to leave space around whole list

\pagestyle{fancy}
\fancyhf{}
\makeatletter
\fancypagestyle{firstpage}{
	\renewcommand\headrulewidth{0pt}
	\fancyhf{}
	\lfoot{\ttfamily\color{SkyBlue}\thepage~|~\@author}
	\rfoot{\ttfamily\color{SkyBlue}\coursenumber~|~\coursequarter}
}
\fancyhead[C]{\ttfamily\color{SkyBlue}\suptitle}
\fancyfoot[L]{\ttfamily\color{SkyBlue}\thepage~|~\@author}
\fancyfoot[R]{\ttfamily\color{SkyBlue}\coursenumber~|~\coursequarter}

\renewcommand\maketitle{
{\raggedright % Note the extra {
\begin{center}
	\ttfamily
	{\large\color{SkyBlue}\suptitle}\\[2ex]
	{\LARGE\@title}\\[2ex]
	{\large\subtitle}\\[2ex]\hrule
\end{center}}} % Note the extra }
\makeatother

\renewcommand\suptitle{Course Syllabus}
\title{\coursenumber~(\coursetitle)}
\renewcommand\subtitle{\coursequarter~\textbullet~\coursedate}

\titleformat{\section}[runin]{\normalfont\sffamily\bfseries}{}{0em}{}[\hangindent=\the\titlewidth]
\titleformat{\subsection}[runin]{\normalfont\sffamily}{\textbullet~}{0em}{}[\hangindent=\the\titlewidth]

	
\renewcommand\coursenumber{Math 110}
\renewcommand\coursetitle{Introduction to Number Theory}
\renewcommand\coursequarter{Fall 2022}
\renewcommand\coursedate{MWF 2:40 PM -- 3:45 PM}
\author{Xu Gao}


% LTeX: enabled=true
\begin{document}
\thispagestyle{firstpage}
\maketitle

\section{Instructor} 
Xu Gao (\email{xgao26@ucsc.edu}), McHenry Library 1292

\section{Lectures}  
The lectures will be \emph{in-person} and will take place at \emph{Soc Sci 2 071} every Monday, Wednesday, and Friday 2:40 PM -- 3:45 PM. 

\section{Office Hours}
Monday and Wednesday 4:30 -- 5:30 PM, or by appointment. 

\noindent
{\footnotesize (This is when you come and talk to me if you have any questions. Do not hesitate to make appointments with me if regular office hours don't work for you!)}

\section{Teaching Assistant} 
Yuk Shing Lam (\email{ylam14@ucsc.edu})

\section{Textbook} We will follow \textit{\color{MidnightBlue} An Illustrated Theory of Numbers} by Martin H Weissman, focusing on Chapters 1 - 8. See \href{http://illustratedtheoryofnumbers.com/}{the book page} for more information on this book. 

\section{Course Schedule} 
The following is a tentative course schedule. 
\begin{center}
	\begin{tabular}{l|m{0.4\textwidth}|l}
		\hline
		\multicolumn{1}{c}{\textsf{Week}} & \multicolumn{1}{c}{\textsf{Topic}} & \multicolumn{1}{c}{\textsf{Assignments}} \\
		\hline
		Week 1 (Sep. 30) & The Euclidean algorithm & HW 1 \\
		\hline
		Week 2 (Oct. 7) & Prime factorization & HW 2, Glossary A \\
		\hline
		Week 3 (Oct. 14) & \multirow{2}{0.4\textwidth}{Rational numbers and algebraic numbers} & HW 3 \\
		\cline{1-1}\cline{3-3}
		Week 4 (Oct. 21) & & HW 4, Glossary B \\
		\hline
		Week 5 (Oct. 28) & \multirow{3}{0.4\textwidth}{The modular worlds and Modular dynamics} & HW 5, Midterm \\
		\cline{1-1}\cline{3-3}
		Week 6 (Nov. 4) & & HW 6 \\
		\cline{1-1}\cline{3-3}
		Week 7 (Nov. 11) & & HW 7, Glossary C \\
		\hline
		Week 8 (Nov. 18) & Assembling the Modular World & HW 8 \\
		\hline
		Week 9 (Nov. 25) & \multirow{2}{0.4\textwidth}{Quadratic residues} & HW 9 \\
		\cline{1-1}\cline{3-3}
		Week 10 (Dec. 2) & & HW 10, Glossary D \\
		\hline
	\end{tabular}
\end{center}

\section{Learning Outcomes}
\begin{itemize}
	\item Familiarize Ideas and problems in number theory that play essential roles in modern mathematics.
	\item Understand the roles of theorems, proofs, and counterexamples. 
	\item Develop problem-solving skills.
	\item Practice clear, concise, and precise mathematical writing.
\end{itemize}	

\section{Grade} 
The grade will be based on: Quizzes (10\%), Glossary
(10\%), Homework (30\%), Midterm (20\%), and Final (30\%). 
We will use the following grading scheme (but curve may apply).
\begin{center}
	\begin{tabular}{||C{0.2\textwidth}|C{0.1\textwidth}||}
		\hline 
		Total scores & Grade \\
		\hline 
		$\geqslant 98$ & A$+$ \\
		$90$ -- $97$ & A \\
		$88$ -- $89$ & A$-$ \\
		\hline 
		$85$ -- $87$ & B$+$ \\
		$78$ -- $84$ & B \\
		$75$ -- $77$ & B$-$ \\
		\hline 
		$70$ -- $74$ & C$+$ \\
		$60$ -- $69$ & C \\
		\hline 
		$55$ -- $59$ & C$-$ \\
		$40$ -- $54$ & D \\
		$< 40$ & F \\
		\hline 
	\end{tabular}
\end{center}

To pass the course, your grade should be at least C.

\section{Guidelines}
\subsection{Quizzes}
\begin{enumerate}
	\item Prepare a draft paper at the beginning of each meeting and write down your \emph{name} and \emph{student ID} on it. 
	\item There may or may not be a quiz near the end of the meeting. If there is, you are expected to finish it in \emph{at most 5 minutes}, and I will show you the answer after that. 
	\item The quizzes will be collected \emph{when the meeting ends}. Even if there is no quiz in one meeting, I will still collect the draft paper with your name and student ID as a participation record. 
	\item To obtain the full grade, one need to attend at least 25 meetings. 
\end{enumerate}

\subsection{Glossary}
\begin{enumerate}
	\item Throughout the course, you will maintain a glossary of terms and results that you find difficult to digest or wish to remember. Add \emph{your thoughts} on them, and whenever possible, include examples as well.
	\item The glossary can be typed or handwritten, long or short, but it \emph{cannot be empty}. The point is it is about your learning experience. \href{https://everydaymath.uchicago.edu/teachers/TRM-Glossary-G4-6_correct.pdf}{This PDF file} may give an idea of what a glossary looks like. 
	\item You are asked to share your glossary every two or three weeks. To do this, navigate to the Glossary page and upload a \emph{PDF} file to Gradescope. Be aware of the \emph{due date}. 
\end{enumerate}

\subsection{Homework}
\begin{enumerate}
	\item There will be homework every week. Please turn in the homework on the \emph{due date}. (If you need an extension, please ask before the due date.)
	\item You are encouraged to \emph{discuss} the problems with your peers (for example, on ED Discuss). However, you must write the homework \emph{by yourself} using your words and \emph{acknowledge your collaborators}. 
	\item The homework is expected to be typed using \emph{\LaTeX}. 
	\item Pay close attention to the presentation and the clarity of your reasoning. This course is writing-intensive.
	\item List the \emph{references} you have used in your answer. You should avoid using resources that solve the problem immediately. 
	\item To submit the homework, navigate to the Homework page and upload the \emph{compiled PDF} file (not the .tex file) to Gradescope. 
	% \item The lowest one will be draopped.
\end{enumerate}

\subsection{Midterm}
\begin{enumerate}
	\item The Midterm will be given during the lecture times. The exact date and time will be posted on Canvas. 
	\item Please prepare draft papers before the exam. 
	\item You can use your notes and textbook during the exam. But you \emph{cannot discuss} the problems with others. 
	\item The only results (theorems/lemmas/propositions) you're allowed to use are either provided during the lectures or in the homework. 
\end{enumerate}

\subsection{Final}
\begin{enumerate}
	\item According to \href{https://registrar.ucsc.edu/soc/final-examinations.html}{the registrar}, the final will be on Tuesday, Dec. 6, from 4:00 PM to 7:00 PM. 
	\item You can look up your notes, homework, and textbook during the exam. But you \emph{cannot discuss} the problems with others. \emph{No digital devices} are allowed during the exam. 
	\item The only results (theorems/lemmas/propositions) you're allowed to use are either provided during the lectures or in the homework.
	\item If you have any questions, ask me or the TA.
\end{enumerate}
	
\subsection{Communication}
\begin{enumerate}
	\item You are welcome to ask me any questions during class and the office hours.
	\item Besides those times, please contact me primarily via \emph{Canvas}. I will reply within 24 hours during the weekdays. Contact me via email if and only if you haven't received any response for a long time.  
	\item When you contact me, please provide as much information as possible on the subject you intend to discuss. 
	\item \emph{Never hesitate to reach out}. 
\end{enumerate}

\section{Ed Discussion}
We will use Ed Discussion as a discussion forum for this course. You are encouraged to collaborate and answer each other's questions. You can access it from the sidebar on Canvas.

\section{\LaTeX}
{\LaTeX} is widely used for mathematical writing. You will need basic {\LaTeX} for this course. To start, look at the .tex files I provide you. You can upload them to online {\LaTeX} editors such as \href{https://www.overleaf.com/}{Overleaf}, or import them into a local editor. 
Then google anything beyond the files, and use the online resources such as \href{https://tex.stackexchange.com/}{Tex Stack Exchange}, \href{https://www.overleaf.com/learn}{Overleaf knowledge base}, and \href{https://en.wikibooks.org/wiki/LaTeX}{{\LaTeX} wikibook}. If you need help getting started, please contact me. 

\section{Academic Integrity} 
All members of the UCSC community benefit from an environment of trust, honesty, fairness, respect, and responsibility. You are expected to present your own work and acknowledge the work of others to preserve the integrity of scholarship. For the full policy and disciplinary procedures on academic dishonesty, refer to the \href{https://ue.ucsc.edu/academic-misconduct.html}{Academic Misconduct page} at the Division of Undergraduate Education. 

\section{Accessibility (DRC)} 
UC Santa Cruz is committed to creating an academic environment that supports its diverse student body. If you are a student with a disability who requires accommodations to achieve equal access in this course, please submit your Accommodation Authorization Letter from the Disability Resource Center (DRC) to me privately during my office hours or by appointment, preferably within the first two weeks of the quarter. At this time, I would also like us to discuss ways we can ensure your full participation in the course. I encourage all students who may benefit from learning more about DRC services to contact the DRC by phone at 831-459-2089 or by email at \email{drc@ucsc.edu}.

\section{CAPS (Counseling and Psychological Services)}
The university offers a variety of confidential services to help you through difficult times, including individual and group counseling, crisis intervention, consultations, online chats, and mental health screenings. See \href{https://caps.ucsc.edu/}{the CAPS page} for more information.

\section{COVID-19 Information for In-person Courses} 
\subsection{WHAT WE CAN EXPECT FROM EACH OTHER}
Each individual at UC Santa Cruz should act in the best interests of everyone else in our community. Please take care to comply with all university guidelines about masking in indoor settings, performing daily symptom and badge checks, testing as required by the campus vaccine policy, self-isolating in the event of exposure, and respecting others' comfort with distancing. Please do not come to class if your badge is not green. If you forget your mask, you can ask me for one; there is a limited supply of disposable masks in each classroom. If you are ill or suspect you may have been exposed to someone who is ill, or if you have symptoms that are in any way similar to those of COVID-19, please err on the side of caution and stay home until you are well or have tested negative after an exposure. Let me know if you're not feeling well, and I'll respond about how best you can keep learning.
 
\subsection{WHAT YOU CAN EXPECT FROM ME}
I have designed our course following campus guidance and with current public health guidelines in mind. However, these guidelines may change in accordance with shifting infection rates or the emergence of new variants. If updated public health recommendations and university requirements make our current course format unfeasible, or if I experience a need to self-isolate, I will alter the format. This may include moving in-person sessions onto Zoom, modifying course assignments to work in a remote format, and reconfiguring exams (if applicable). I will communicate clearly with you via email or Canvas announcement about any changes that occur. I will provide as much advance warning as possible and give you all the information you need to transition smoothly to the new format. If you have questions about the changes, please reach out to me, so I can answer them.
 
\subsection{WHAT I EXPECT OF YOU}
If you experience an illness or exposure that requires you to miss class sessions or to attend remotely, please communicate with me as soon as possible, and I will provide you with options to allow you to continue making progress in the class. 

\section{Title IX and Care Advisory}
UC Santa Cruz is committed to providing a safe learning environment that is free of all forms of gender discrimination and sexual harassment, which are explicitly prohibited under Title IX. If you have experienced any form of sexual harassment, sexual assault, domestic violence, dating violence, or stalking, know that you are not alone. The Title IX Office, the Campus Advocacy, Resources \& Education (CARE) office, and Counseling \& Psychological Services (CAPS) are all resources that you can rely on for support. 

Please be aware that if you tell me about a situation involving Title IX misconduct, I am required to share this information with the Title IX Coordinator. This reporting responsibility also applies to course TAs and tutors (as well as to all UCSC employees who are not designated as ``confidential'' employees, which is a special designation granted to counselors and CARE advocates). Although I must make that notification, you will control how your case will be handled, including whether you wish to pursue a formal complaint. The goal is to make sure that you are aware of the range of options available to you and that you have access to the resources you need. 

Confidential resources are available through \href{https://care.ucsc.edu/}{CARE}. Confidentiality means CARE advocates will not share any information with Title IX, the police, parents, or anyone else without explicit permission. CARE advocates are trained to support you in understanding your rights and options, accessing health, and counseling services, providing academic and housing accommodations, helping with legal protective orders, and more. You can contact CARE at (831) 502-2273 or \email{care@ucsc.edu}.

In addition to CARE, these resources are available to you:
\begin{itemize}
	\item If you need help figuring out what resources you or someone else might need, visit the \href{http://safe.ucsc.edu/}{Sexual Violence Prevention \& Response (SAFE) website}, which provides information and resources for different situations.
	\item \href{https://caps.ucsc.edu/}{Counseling \& Psychological Services (CAPS)} can provide confidential counseling support. Call them at (831) 459-2628.
	\item You can also report gender discrimination and sexual harassment and violence directly to the University's \href{https://titleix.ucsc.edu/}{Title IX Office}, by calling (831) 459-2462 or by using their \href{https://titleix.ucsc.edu/about/staff-contact-us.html}{online reporting tool}.
	\item Reports to law enforcement can be made to the UC Police Department, (831) 459-2231 ext. 1.
	\item For emergencies, call 911.	
\end{itemize}

\section{Report an Incident of Hate / Bias}
The University of California, Santa Cruz is committed to maintaining an objective, civil, diverse, and supportive community, free of coercion, bias, hate, intimidation, dehumanization, or exploitation. The Hate/Bias Response Team is a group of administrators who support and guide students seeking assistance in determining how to handle a bias incident involving another student, a staff member, or a faculty member. To report an incident of hate or bias, please use the following form: \href{https://ucsc-advocate.symplicity.com/care_report/index.php/}{Hate/Bias Report Form}.

\section{Religious Accommodation}
UC Santa Cruz welcomes a diversity of religious beliefs and practices, recognizing the contributions differing experiences and viewpoints can bring to the community. There may be times when an academic requirement conflicts with religious observances and practices. If that happens, students may request reasonable accommodation for religious practices. The instructor will review the situation to provide reasonable accommodation without penalty. You should first discuss the conflict and your requested accommodation with your instructor early in the term. You may also seek assistance from the \href{https://deanofstudents.ucsc.edu/}{Dean of Students office}.

\section{Tutoring and Learning Support}
At \href{https://lss.ucsc.edu/}{Learning Support Services (LSS)}, undergraduate students build a strong foundation for success and cultivate a sense of belonging in our Community of Learners. LSS partners with faculty and staff to advance educational equity by designing inclusive learning environments in Modified Supplemental Instruction, Small Group Tutoring, and Writing Support. When students fully engage in our programs, they gain transformative experiences that empower them at the university and beyond.

\end{document}