%LaTeX Template 
%%	by	Xu Gao (gausyu@gmail.com)
%%LPPL: http://www.latex-project.org/lppl.txt
%------------------------
%The following is the Preamble
%------------------------
\documentclass[11pt]{article}
%This is the document class. 
%%	Most common: article, beamer, book, etc.
%%	11pt means the default font size is 11pt. One can also use 10pt, 11pt, or 12pt.
%%	See https://en.wikibooks.org/wiki/LaTeX/Document_Structure#Document_classes for explaination.
%------------------------
%Packages
%------------------------
\usepackage[a4paper,total={6in, 9in}]{geometry} 
%This aims to customize page layout
\usepackage[T1]{fontenc}
%Using T1 font
\usepackage{mathtools}
%The main MATH package (for whom may wonder, mathtools load amsmath automatically, so no need \usepackage{amsmath})
\usepackage{amsthm}
%This defines the theorem enviroments
\usepackage{amssymb}
%Math symbols
\usepackage{bm}
%Bold math fonts, provide \bm{} command
\usepackage[scr=rsfs,cal=euler]{mathalpha}
%Math fonts 
%%The package provides means of loading maths alphabets (such as are normally addressed via macros \mathcal, \mathbb, \mathfrak and \mathscr)
%%How to use? "scr=" set what font \mathscr use, "cal=" set what font \mathcal use, "bb=" set what font \mathbb use, and "frak=" set what font \mathfrak use.
\usepackage{tikz}
%Drawing
\usepackage{graphicx}
%Support for graphics
\usepackage{hyperref}
\usepackage[nameinlink]{cleveref}
%Hyper links
\hypersetup{
	colorlinks=true,
	linkcolor=blue,
	urlcolor=magenta,
	citecolor=blue
}
%Color links
\usepackage{etoolbox}
%Programming
\usepackage[shortlabels]{enumitem}
%Enable enumerate modification
\usepackage{tcolorbox}
%provides an environment for coloured and framed text boxes with a heading line.
\tcbset{colback=white}
%------------------------
%Theorem-like Enviroments
%------------------------
\theoremstyle{plain}
%This is the theorem style, plain means blod title and italic body
\newtheorem{theorem}{Theorem}[section]
%This is how you define a theorem-like enviroment.
\newtheorem{lemma}[theorem]{Lemma}
\newtheorem{proposition}[theorem]{Proposition}
\newtheorem{corollary}[theorem]{Corollary}
%The option [equation] means the lemmas are numbered following the same system of equation.
\newtheorem*{propstar}{Proposition}
%Star version defines unnumbered enviroments
\theoremstyle{definition}
%This is the theorem style, definition means blod title and normal body
\newtheorem{definition}[theorem]{Definition}
\newtheorem{example}[theorem]{Example}
%The option [equation] means the examples are numbered following the same system of equation.
\newtheorem{problem}{Problem}
\Crefname{problem}{problem}{problems}
\Crefname{problem}{Problem}{Problems}
%This is the enviroment used as problems in HWs.
\theoremstyle{remark}
%This is the theorem style, remark means italic title and normal body
\newtheorem*{remark}{Remark}
\newtheorem*{hint}{Hint}
\newtheorem*{optional}{Optional}
%Remark enviroment
\numberwithin{equation}{section}
%Let equations be numbered with in each problems.
\NewDocumentEnvironment{solution} {o+b} 
{\begin{proof}[\IfNoValueTF{#1}{\textbf{Solution}}{\textbf{Solution} (#1)}]#2\end{proof}}{}
%This defines the enviroment for you to write solutions.
%------------------------
\newlist{listinprob}{enumerate}{1}
\setlist[listinprob]{label=(\alph{listinprobi}),
									ref=\theproblem.(\alph{listinprobi}),
									noitemsep}
\Crefname{listinprobi}{problem}{problems}
\Crefname{listinprobi}{Problem}{Problems}
%This defines a new list type used in the enviroment problem.
%------------------------
%DocumentCommands
%------------------------
%Notations for number sets
\NewDocumentCommand \N {} { \mathbb{N} }%Natural numbers
\NewDocumentCommand \Z {} { \mathbb{Z} }%Integers
\NewDocumentCommand \Q {} { \mathbb{Q} }%Rational Numbers
\NewDocumentCommand \R {} { \mathbb{R} }%Real Numbers
\NewDocumentCommand \C {} { \mathbb{C} }%Complex Numbers
\NewDocumentCommand \F {} { \mathbb{F} }%Field
%Notations for maps
\DeclareMathOperator{\id}{id} % identity
\DeclareMathOperator{\pr}{pr} % projection
\DeclareMathOperator{\pt}{pt} % point
\DeclareMathOperator{\res}{res} % restriction
%PairedDelimiters
\DeclarePairedDelimiterX\abs[1]\lvert\rvert
	{ \ifblank{#1}{\:\cdot\:}{#1} }
%abstract value function
\DeclarePairedDelimiterX\norm[1]\lVert\rVert
	{ \ifblank{#1}{\:\cdot\:}{#1} }
%the norm function
\DeclarePairedDelimiterX\ceil[1]\lceil\rceil
	{ \ifblank{#1}{\:\cdot\:}{#1} }
%ceil function
\DeclarePairedDelimiterX\floor[1]\lfloor\rfloor
	{ \ifblank{#1}{\:\cdot\:}{#1} }
%floor function
\DeclarePairedDelimiterX\pairing[2]\langle\rangle
	{ \ifblank{#1}{\:\cdot\:}{#1}, \ifblank{#2}{\:\cdot\:}{#2} }
\DeclarePairedDelimiterX\inner[2]\lparen\rparen
	{ \ifblank{#1}{\:\cdot\:}{#1}, \ifblank{#2}{\:\cdot\:}{#2} }
%Inner product
\providecommand\given{}
\newcommand\SetSymbol[1][]
	{ \nonscript\:#1\vert\allowbreak\nonscript\:\mathopen{} }
\DeclarePairedDelimiterX\Set[1]\{\}
	{ \renewcommand\given{\SetSymbol[\delimsize]}#1 }
%a set
%
%Miscellaneous
\NewDocumentCommand \vect { m } { \mathbf{#1} }
%Vector
\RenewDocumentCommand \le {} { \leqslant }
\RenewDocumentCommand \ge {} { \geqslant }
%Change the inequality symbols
\NewDocumentCommand \txforall {} {\quad\text{for all}\quad}
%------------------------
% Additional math symbols, used for the Math 110 course
% \DeclareMathOperator*\GCD{GCD}
\DeclareMathOperator*\lcm{lcm}
%------------------------
\allowdisplaybreaks
%Allow a long equation to display in more than one page.

%The following change the default layout of title
\makeatletter
\def\@maketitle{%
	\newpage
	\null
	\vskip 2em%
	\begin{center}%
		\let \footnote \thanks
		\sffamily 
		{\LARGE \@title \par}%
		\vskip 1.5em%
		{\large \@subtitle \par}%
		\vskip 1.5em%
		{\large Student name: \@author \par}%
		\ifdefempty{\@acknowledge}{}%
		{\large With help of: \@acknowledge \par}%
		\vskip 1em%
		{\large \@date}%
	\end{center}%
	\par
	\vskip 1.5em%
}	
\global\let\@subtitle\@empty
\DeclareRobustCommand*{\subtitle}[1]{\gdef\@subtitle{#1}}
\global\let\@acknowledge\@empty
\DeclareRobustCommand*{\acknowledge}[1]{\gdef\@acknowledge{#1}}
\makeatother
%Leave this change alone if you don't know what it means

%------------------------
%Information of the file
%------------------------
\title{Möbius Inversion Formula}
%The title of the essay
\author{[Your Full Name Here]}
%Input your full name here.
% \acknowledge{[Your collaborators]}
%Input your collaborators, if none, comment it.
\subtitle{MATH 110~|~Introduction to Number Theory~|~Summer 2023}
%The subtitle
\date{\today} 
%The date, if commented, the value will be \today

%------------------------
%Document starts here
%------------------------
\begin{document}
\maketitle

Arithmetic functions play a crucial role in number theory, providing insights into the fundamental properties of integers. 
The \emph{Möbius inversion formula}, introduced into number theory in 1832 by August Ferdinand Möbius, serves as a bridge between different arithmetic functions, allowing us to deduce properties and reveal hidden relationships between them. 
In this essay, we will delve into the world of arithmetic functions, introducing relevant concepts and providing a proof of the Möbius inversion formula.

This essay is organized as follows: 
In \Cref{sec:1}, we will introduce the statement of the formula. 
Moving on to \Cref{sec:2}, we will delve into the concept of \emph{Dirichlet convolution} and explore its fundamental properties, providing the necessary groundwork for the subsequent proof.
In the conclusive \Cref{sec:3}, we will utilize Dirichlet convolution to present a proof of the Möbius inversion formula. 
Finally, in \Cref{sec:4}, we will demonstrate a practical application of the Möbius inversion formula by showcasing its usage on the \emph{Euler totient function}.

\section{The Möbius function and the inversion formula}\label{sec:1}

To state the Möbius inversion formula, we need the following notions.
\begin{definition}
	We say that a positive integer $n$ is \emph{square-free} if $n$ is not divisible by $p^2$ for any prime number $p$.  
	The \emph{Möbius function} is defined as follows:
	\[
		\mu(n) := 
		\begin{dcases*}
			1 & if $n=1$,\\
			0 & if $n$ is not sqaure-free,\\
			(-1)^{t} & if $n$ is sqaure-free and has exactly $t$ prime divisors.
		\end{dcases*}
	\]
\end{definition}
The Möbius function $\mu$ is an \emph{arithmetic function}. Namely, it is a complex-valued function defined on the set $\Z_{+}$ of positive integers.
\begin{definition}
	The \emph{Möbius transformation} of an arithmetic function $f$ is the function $\widehat{f}$ defined by the formula 
	\begin{equation}\label{eq:1}
		\widehat{f}(n) := \sum_{d\mid n} f(d).
	\end{equation} 
\end{definition}

Then the Möbius inversion formula can be stated as follows.
\begin{theorem}[Möbius inversion formula]\label{thm:mif}
	For any arithmetic function $f$, we have 
	\begin{equation}\label{eq:2}
		f(n) = \sum_{d\mid n} \mu(\frac{n}{d})\widehat{f}(d).
	\end{equation}
\end{theorem}

\section{Dirichlet convolution}\label{sec:2}
To better understand the Möbius inversion formula, we introduce the following concept.
\begin{definition}
	Given two arithmetic functions $f$ and $g$, their \emph{Dirichlet convolution}, denoted as $f \star g$, is defined as follows:
	\begin{equation} 
		(f \star g)(n) = \sum_{d|n} f(d)g\left(\frac{n}{d}\right) 
	\end{equation}
	where the summation is taken over the divisor set $\mathscr{D}(n):=\Set*{ d \given d \text{ is a divisor of } n }$. 
\end{definition}

The basic property of Dirichlet convolution is the following.
\begin{theorem}
	The set of arithmetic functions, together with the Dirichlet convolution, forms a commutative monoid. 
	Namely, the followings hold.
	\begin{enumerate}
		\item The binary operation $\star$ is \emph{associative}: for any arithmetic functions $f$, $g$, and $h$, 
		\begin{equation}
			(f\star g)\star h = f\star (g\star h).
		\end{equation}
		\item The binary operation $\star$ has a \emph{neutral element}. 
		Indeed, let $\delta$ be the function defined as follows:
		\begin{equation*}
			\delta(n) := 
			\begin{dcases*}
				1 & if $n=1$,\\
				0 & if otherwise.
			\end{dcases*}
		\end{equation*}
		Then $\delta$ is a \emph{neutral element} for the binary operation $\star$: for any arithmetic function $f$, 
		\begin{equation}
			\delta\star f = f\star\delta = f.
		\end{equation}
		\item The binary operation $\star$ is \emph{commutative}: for any arithmetic functions $f$ and $g$, 
		\begin{equation}
			f\star g = g\star f.
		\end{equation}
	\end{enumerate}
\end{theorem}
\begin{proof}
	{\color{red} (PROOF NEED TO BE FILLED)}
\end{proof}

Among all arithmetic functions, the \emph{multiplicative} one usually play special roles. 
\begin{definition}
	An arithmetic function $f$ is said to be \emph{multiplicative}, if for any pair of coprime positive integers $(m,n)$, we have 
	\[
		f(mn) = f(m)f(n).
	\]
\end{definition}
We can restrict Dirichlet convolution on the subset of multiplicative functions.
\begin{proposition}
	The subset of multiplicative functions, together with the Dirichlet convolution, forms a submonoid of the commutative monoid of all arithmetic functions. Namely, we have:
	\begin{enumerate}
		\item The subset of multiplicative functions is closed under the binary operation $\star$: for any arithmetic functions $f$ and $g$, if both $f$ and $g$ are multiplicative, then $f\star g$ is also multiplicative.
		\item The subset of multiplicative functions contains the neutral element $\delta$ of the binary operation $\star$.
	\end{enumerate}
\end{proposition}
\begin{proof}
	{\color{red} (PROOF NEED TO BE FILLED)}
\end{proof}

It is worth to note that
\begin{proposition}
	The Möbius function $\mu$ is multiplicative.
\end{proposition}
\begin{proof}
	{\color{red} (PROOF NEED TO BE FILLED)}
\end{proof}

\section{Proof of the Möbius Inversion Formula}\label{sec:3}
With the foundation of Dirichlet convolution in place, we are ready to present the proof of the Möbius inversion formula. 

We first note that 
\begin{lemma}\label{lem:mi}
	For any arithmetic function $f$, its M\"obius transformation $\widehat{f}$ is $f\star\bm{1}$. 
	Where $\bm{1}$ is the constant function mapping any positive integer to $1$. 
\end{lemma}
\begin{proof}
	{\color{red} (PROOF NEED TO BE FILLED)}
\end{proof}
\begin{corollary}\label{cor:mtism}
	The M\"obius transformation of a multiplicative function is also multiplicative.
\end{corollary}
\begin{proof}
	{\color{red} (PROOF NEED TO BE FILLED)}
\end{proof}

Moreover, we have 
\begin{lemma}\label{lem:muone=delta}
	The Möbius function $\mu$ is the inverse of $\bm{1}$ under the binary operation $\star$. 
	Namely, $\bm{1}\star\mu = \delta$.
\end{lemma}
\begin{proof}
	{\color{red} (PROOF NEED TO BE FILLED)}
\end{proof}

Now, we can restate \Cref{thm:mif} using Dirichlet convolution.
\begin{theorem}\label{thm:mif-2}
	For any arithmetic function $f$, we have 
	\begin{equation}\label{eq:thm9}
		f = \widehat{f}\star\mu.
	\end{equation}
\end{theorem}
Indeed, \Cref{eq:2,eq:thm9} are equivalent: {\color{red} (PROOF NEED TO BE FILLED)}

We are now able to demonstrate the proof of \Cref{thm:mif-2}.
\begin{proof}
	We have 
	\begin{align*}
		\widehat{f}\star\mu 
		&= (f\star\bm{1})\star\mu &\text{by \Cref{lem:mi}}\\
		&= f\star(\bm{1}\star\mu) &\text{by associativity of $\star$}\\
		&= f\star\delta &\text{by \Cref{lem:muone=delta}}\\
		&= f &\text{$\delta$ is the neutral element for $\star$}.
	\end{align*}
	This proves \Cref{eq:thm9}.
\end{proof}

We are thus able to extend \Cref{cor:mtism}:
\begin{corollary}\label{cor:mitism}
	An arithmetic function $f$ is multiplicative if and only if its M\"obius transformation $\widehat{f}$ is multiplicative.
\end{corollary}
\begin{proof}
	{\color{red} (PROOF NEED TO BE FILLED)}
\end{proof}


\section{Euler totient function}\label{sec:4}
We will use the Möbius inversion formula to obtain some properties of the \emph{Euler totient function}.
\begin{definition}
	The \emph{Euler totient function} $\varphi(n)$ counts the set $\Phi(n)$ of integers from $0$ to $n-1$ which are coprime to $n$.
\end{definition}

Our key step is:
\begin{theorem}
	Let $\id$ denote the identity map.
	Then $\id$ is the Möbius transformation of the Euler totient function $\varphi$.
\end{theorem}
\begin{proof}
	{\color{red} (PROOF NEED TO BE FILLED)}
\end{proof}

Then we have the followings.
\begin{corollary}
	The Euler totient function $\varphi$ is multiplicative.
\end{corollary}
\begin{proof}
	First note that $\id$ is multiplicative: {\color{red} (PROOF NEED TO BE FILLED)}.

	Since $\id$ is the Möbius transformation of $\varphi$, by \Cref{cor:mitism}, $\varphi$ is also multiplicative. 
\end{proof}

\begin{corollary}
	We have the following formula.
	\begin{equation}\label{eq:totient}
		\varphi(n) = 
		n
		\prod_{\text{$p$ is a prime divisor of $n$}}
		\left(1-\frac{1}{p}\right).
	\end{equation}
\end{corollary}
\begin{proof}
	Suppose the prime factorization of $n$ gives 
	\[
		n = \prod_{\text{$p$ is a prime divisor of $n$}}p^{e_p}.
	\]
	Then by the multiplicativity of $\varphi$, we have 
	\[
		\varphi(n) = \prod_{\text{$p$ is a prime divisor of $n$}}\varphi(p^{e_p}).
	\]
	On the other hand, the right-hand side of \Cref{eq:totient} can be written as 
	\[
		\prod_{\text{$p$ is a prime divisor of $n$}}\left(p^{e_p}1-p^{e_p-1}\right).
	\]
	Hence, it suffices to proof the following lemma:
	\begin{lemma}
		For any prime $p$ and any positive integer $e$, we have $\varphi(p^e)=p^e-p^{e-1}$.
	\end{lemma}
	\begin{proof}
		{\color{red} (PROOF NEED TO BE FILLED)}
	\end{proof}
	Now \Cref{eq:totient} follows by previous argument.
\end{proof}


%%%	List your reference here
%------------------------
\begin{thebibliography}{9}  %Do not delete

\bibitem[TEXT]{texbook}
\emph{An Illustrated Theory of Numbers}, Martin H. Weissman. % this is our textbook

% Modify as needed

\end{thebibliography}  %Do not delete
%------------------------
\end{document}
