%LaTeX Template 
%%	by	Xu Gao (gausyu@gmail.com)
%%LPPL: http://www.latex-project.org/lppl.txt
%------------------------
%The following is the Preamble
%------------------------
	\documentclass[11pt]{article}
	%This is the document class. 
	%%	Most common: article, beamer, book, etc.
	%%	11pt means the default font size is 11pt. One can also use 10pt, 11pt, or 12pt.
	%%	See https://en.wikibooks.org/wiki/LaTeX/Document_Structure#Document_classes for explaination.
	%------------------------
	%Packages
	%------------------------
	\usepackage[a4paper,total={6in, 9in}]{geometry} 
	%This aims to customize page layout
	\usepackage[T1]{fontenc}
	%Using T1 font
	\usepackage{mathtools}
	%The main MATH package (for whom may wonder, mathtools load amsmath automatically, so no need \usepackage{amsmath})
	\usepackage{amsthm}
	%This defines the theorem enviroments
	\usepackage{amssymb}
	%Math symbols
	\usepackage{bm}
	%Bold math fonts, provide \bm{} command
	\usepackage[scr=rsfs,cal=euler]{mathalpha}
	%Math fonts 
	%%The package provides means of loading maths alphabets (such as are normally addressed via macros \mathcal, \mathbb, \mathfrak and \mathscr)
	%%How to use? "scr=" set what font \mathscr use, "cal=" set what font \mathcal use, "bb=" set what font \mathbb use, and "frak=" set what font \mathfrak use.
	\usepackage{tikz}
	%Drawing
	\usepackage{graphicx}
	%Support for graphics
	\usepackage{hyperref}
	\usepackage[nameinlink]{cleveref}
	%Hyper links
	\hypersetup{
		colorlinks=true,
		linkcolor=blue,
		urlcolor=magenta,
		citecolor=blue
	}
	%Color links
	\usepackage{etoolbox}
	%Programming
	\usepackage[shortlabels]{enumitem}
	%Enable enumerate modification
	\usepackage{tcolorbox}
	%provides an environment for coloured and framed text boxes with a heading line.
	\tcbset{colback=white}
	%------------------------
	%Theorem-like Enviroments
	%------------------------
	\theoremstyle{plain}
	%This is the theorem style, plain means blod title and italic body
	\newtheorem{theorem}{Theorem}
	%This is how you define a theorem-like enviroment.
	\newtheorem{lemma}[equation]{Lemma}
	\newtheorem{proposition}[equation]{Proposition}
	%The option [equation] means the lemmas are numbered following the same system of equation.
	\newtheorem*{propstar}{Proposition}
	%Star version defines unnumbered enviroments
	\theoremstyle{definition}
	%This is the theorem style, definition means blod title and normal body
	\newtheorem{example}[equation]{Example}
	%The option [equation] means the examples are numbered following the same system of equation.
	\newtheorem{problem}{Problem}
	\crefname{problem}{problem}{problems}
	\Crefname{problem}{Problem}{Problems}
	%This is the enviroment used as problems in HWs.
	\theoremstyle{remark}
	%This is the theorem style, remark means italic title and normal body
	\newtheorem*{remark}{Remark}
	\newtheorem*{hint}{Hint}
	\newtheorem*{optional}{Optional}
	%Remark enviroment
	\numberwithin{equation}{problem}
	%Let equations be numbered with in each problems.
	\NewDocumentEnvironment{solution} {o+b} 
	{\begin{proof}[\IfNoValueTF{#1}{\textbf{Solution}}{\textbf{Solution} (#1)}]#2\end{proof}}{}
	%This defines the enviroment for you to write solutions.
	%------------------------
	\newlist{listinprob}{enumerate}{1}
	\setlist[listinprob]{label=(\alph{listinprobi}),
										ref=\theproblem.(\alph{listinprobi}),
										noitemsep}
	\crefname{listinprobi}{problem}{problems}
	\Crefname{listinprobi}{Problem}{Problems}
	%This defines a new list type used in the enviroment problem.
	%------------------------
	%DocumentCommands
	%------------------------
	%Notations for number sets
	\NewDocumentCommand \N {} { \mathbb{N} }%Natural numbers
	\NewDocumentCommand \Z {} { \mathbb{Z} }%Integers
	\NewDocumentCommand \Q {} { \mathbb{Q} }%Rational Numbers
	\NewDocumentCommand \R {} { \mathbb{R} }%Real Numbers
	\NewDocumentCommand \C {} { \mathbb{C} }%Complex Numbers
	\NewDocumentCommand \F {} { \mathbb{F} }%Field
	%Notations for maps
	\DeclareMathOperator{\id}{id} % identity
	\DeclareMathOperator{\pr}{pr} % projection
	\DeclareMathOperator{\pt}{pt} % point
	\DeclareMathOperator{\res}{res} % restriction
	%PairedDelimiters
	\DeclarePairedDelimiterX\abs[1]\lvert\rvert
		{ \ifblank{#1}{\:\cdot\:}{#1} }
	%abstract value function
	\DeclarePairedDelimiterX\norm[1]\lVert\rVert
		{ \ifblank{#1}{\:\cdot\:}{#1} }
	%the norm function
	\DeclarePairedDelimiterX\ceil[1]\lceil\rceil
		{ \ifblank{#1}{\:\cdot\:}{#1} }
	%ceil function
	\DeclarePairedDelimiterX\floor[1]\lfloor\rfloor
		{ \ifblank{#1}{\:\cdot\:}{#1} }
	%floor function
	\DeclarePairedDelimiterX\pairing[2]\langle\rangle
		{ \ifblank{#1}{\:\cdot\:}{#1}, \ifblank{#2}{\:\cdot\:}{#2} }
	\DeclarePairedDelimiterX\inner[2]\lparen\rparen
		{ \ifblank{#1}{\:\cdot\:}{#1}, \ifblank{#2}{\:\cdot\:}{#2} }
	%Inner product
	\providecommand\given{}
	\newcommand\SetSymbol[1][]
		{ \nonscript\:#1\vert\allowbreak\nonscript\:\mathopen{} }
	\DeclarePairedDelimiterX\Set[1]\{\}
		{ \renewcommand\given{\SetSymbol[\delimsize]}#1 }
	%a set
	%
	%Miscellaneous
	\NewDocumentCommand \vect { m } { \mathbf{#1} }
	%Vector
	\RenewDocumentCommand \le {} { \leqslant }
	\RenewDocumentCommand \ge {} { \geqslant }
	%Change the inequality symbols
\NewDocumentCommand \txforall {} {\quad\text{for all}\quad}
	%------------------------
	% Additional math symbols, used for the Math 110 course
% \DeclareMathOperator*\GCD{GCD}
\DeclareMathOperator*\lcm{lcm}
	%------------------------
	\allowdisplaybreaks
	%Allow a long equation to display in more than one page.

	%The following change the default layout of title
	\makeatletter
	\def\@maketitle{%
		\newpage
		\null
		\vskip 2em%
		\begin{center}%
			\let \footnote \thanks
			\sffamily 
			{\LARGE \@title \par}%
			\vskip 1.5em%
			{\large \@subtitle \par}%
			\vskip 1.5em%
			{\large Student name: \@author \par}%
			\ifdefempty{\@acknowledge}{}%
			{\large With help of: \@acknowledge \par}%
			\vskip 1em%
			{\large \@date}%
		\end{center}%
		\par
		\vskip 1.5em%
	}	
	\global\let\@subtitle\@empty
	\DeclareRobustCommand*{\subtitle}[1]{\gdef\@subtitle{#1}}
	\global\let\@acknowledge\@empty
	\DeclareRobustCommand*{\acknowledge}[1]{\gdef\@acknowledge{#1}}
	\makeatother
%Leave this change alone if you don't know what it means

%------------------------
%Information of the file
%------------------------
\title{Homework 2}
%The title of the file
\author{[Your Full Name Here]}
%Input your full name here.
% \acknowledge{[Your collaborators]}
%Input your collaborators, if none, comment it.
\subtitle{MATH 110~|~Introduction to Number Theory~|~Summer 2023}
%The subtitle
\date{\today} 
%The date, if commented, the value will be \today
 
%------------------------
%Document starts here
%------------------------
\begin{document}
\maketitle

\begin{quotation}
	Whenever you use a result or claim a statement, provide a \textbf{justification} or a \textbf{proof}, unless it has been covered in the class. In the later case, provide a \textbf{citation} (such as ``by the \emph{2-out-of-3 principle}'', ``by Coro. 0.31 in the textbook'', or ``by \cite[Coro. 0.31]{texbook}'').

	You are encouraged to \emph{discuss} the problems with your peers. However, you must write the homework \textbf{by yourself} using your words and \textbf{acknowledge your collaborators}.
\end{quotation}



\begin{problem}
	Prove that if $n$ is a positive integer, and $\sigma_0(n)$ is prime then $n$ is a power of a prime number.
\end{problem}
%----------------------------------------
\begin{solution} %Do not delete
WRITE YOUR SOLUTION HERE
\end{solution}\clearpage %Do not delete
%----------------------------------------

\begin{problem}[Mersenne, 1644]
	Describe all circumstances under which $\sigma_1(n)$ is odd.
	\begin{hint}
		Consider the prime factorization of $n$.
	\end{hint}
\end{problem}
%----------------------------------------
\begin{solution} %Do not delete
WRITE YOUR SOLUTION HERE
\end{solution}\clearpage %Do not delete
%----------------------------------------



\begin{problem}
	Recall that an \emph{integer polynomial} is an expression of the form 
	\[P(T)=c_dT^d+\cdots+c_1T+c_0,\] 
	where each $c_i$ is an integer. 
	\begin{listinprob}
		\item\label{RN:p1.a} \textbf{Find} a nonzero integer polynomial $P(T)$ that has $\sqrt{3}+\sqrt[3]{5}$ as a root.
	\end{listinprob}
%----------------------------------------
\begin{solution} %Do not delete
WRITE YOUR SOLUTION HERE
\end{solution}\clearpage %Do not delete
%----------------------------------------
	\begin{listinprob}[resume]
		\item \textbf{Prove that} $\sqrt{3}+\sqrt[3]{5}$ is irrational using \ref{RN:p1.a}.
	\end{listinprob}
%----------------------------------------
\begin{solution} %Do not delete
WRITE YOUR SOLUTION HERE
\end{solution}\clearpage %Do not delete
%----------------------------------------
\end{problem}


\begin{problem}
	By evaluating the Taylor series for the exponential function:
	\[
		e^{x} = 1 + \frac{x}{1!} + \frac{x^2}{2!} + \cdots + \frac{x^n}{n!} + \cdots
	\]
	at $x=1$, we get the formula
	\[
		e = 1 + \frac{1}{1!} + \frac{1}{2!} + \frac{1}{3!} + \cdots + \frac{1}{n!} + \cdots.
	\]
	In this problem, you will prove that $e$ is \emph{irrational}.
	\begin{listinprob}
		\item Let $s_n := \sum\limits_{k=0}^n \frac{1}{k!}$, the $n$-th partial sum of above series. \textbf{Show that} 
		\[
			0 \leq e - s_n \leq \frac{1}{n}\cdot \frac{1}{n!}.
		\]
	\end{listinprob}
%----------------------------------------
\begin{solution} %Do not delete
WRITE YOUR SOLUTION HERE
\end{solution}\clearpage %Do not delete
%----------------------------------------
	\begin{listinprob}[resume]
		\item Assume $e$ is rational, and say $a/b$ is the reduced fraction representing $e$. Apply the previous result to $n = b$ and arrive at a contradiction.
	\end{listinprob}
%----------------------------------------
\begin{solution} %Do not delete
WRITE YOUR SOLUTION HERE
\end{solution}\clearpage %Do not delete
%----------------------------------------
\end{problem}

\begin{problem}
	Consider the \emph{Fibonacci numbers}, define recursively by
	\[
		F_0 = 0, F_1 = 1, \text{ and } F_n = F_{n-1} + F_{n-2} \txforall 
		n\geq 2;
	\]
	so the first few terms are 
	\[
		0,1,1,2,3,5,8,13,\cdots.
	\]
	For all $n\geq 2$, define the rational number $r_n$ by the fraction $\dfrac{F_n}{F_{n-1}}$; so the first few terms are
	\[
		\frac{1}{1}, \frac{2}{1}, \frac{3}{2}, \frac{5}{3}, \frac{8}{5}, \cdots.
	\]
	\begin{listinprob}
		\item Prove that for all $n\geq 4,$ we have $r_n = r_{n-1} \vee r_{n-2}$.
	\end{listinprob}
%----------------------------------------
\begin{solution} %Do not delete
WRITE YOUR SOLUTION HERE
\end{solution}\clearpage %Do not delete
%----------------------------------------
	\begin{listinprob}[resume]
		\item Prove that the sequence $r_n$ converges (to a real number).
	\end{listinprob}
%----------------------------------------
\begin{solution} %Do not delete
WRITE YOUR SOLUTION HERE
\end{solution}\clearpage %Do not delete
%----------------------------------------
	\begin{listinprob}[resume]
		\item Prove that $r_n$ converges to the \emph{golden ratio}:
		\[
			\phi = \frac{1 + \sqrt{5}}{2}.
		\]
	\end{listinprob}
	For this problem, you can use any result that you may have seen in your Calculus classes.
%----------------------------------------
\begin{solution} %Do not delete
WRITE YOUR SOLUTION HERE
\end{solution}\clearpage %Do not delete
%----------------------------------------
\end{problem}


%%%	List your reference here
%------------------------
\begin{thebibliography}{9}  %Do not delete
%List your references here
\bibitem[TEXT]{texbook}
\emph{An Illustrated Theory of Numbers}, Martin H. Weissman.

\bibitem[label]{citekey}
[Book(s): \emph{Title}, Author ] or [Online: \href{http://example.com/}{Link}]
\end{thebibliography}  %Do not delete
%------------------------
\end{document}
