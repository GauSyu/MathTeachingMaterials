%LaTeX Template 
%%	by	Xu Gao (gausyu@gmail.com)
%%LPPL: http://www.latex-project.org/lppl.txt
%------------------------
%The following is the Preamble
%------------------------
	\documentclass[11pt]{article}
	%This is the document class. 
	%%	Most common: article, beamer, book, etc.
	%%	11pt means the default font size is 11pt. One can also use 10pt, 11pt, or 12pt.
	%%	See https://en.wikibooks.org/wiki/LaTeX/Document_Structure#Document_classes for explaination.
	%------------------------
	%Packages
	%------------------------
	\usepackage[a4paper,total={6in, 9in}]{geometry} 
	%This aims to customize page layout
	\usepackage[T1]{fontenc}
	%Using T1 font
	\usepackage{mathtools}
	%The main MATH package (for whom may wonder, mathtools load amsmath automatically, so no need \usepackage{amsmath})
	\usepackage{amsthm}
	%This defines the theorem enviroments
	\usepackage{amssymb}
	%Math symbols
	\usepackage{bm}
	%Bold math fonts, provide \bm{} command
	\usepackage[scr=rsfs,cal=euler]{mathalpha}
	%Math fonts 
	%%The package provides means of loading maths alphabets (such as are normally addressed via macros \mathcal, \mathbb, \mathfrak and \mathscr)
	%%How to use? "scr=" set what font \mathscr use, "cal=" set what font \mathcal use, "bb=" set what font \mathbb use, and "frak=" set what font \mathfrak use.
	\usepackage{tikz}
	%Drawing
	\usepackage{graphicx}
	%Support for graphics
	\usepackage{hyperref}
	\usepackage[nameinlink]{cleveref}
	%Hyper links
	\hypersetup{
		colorlinks=true,
		linkcolor=blue,
		urlcolor=magenta,
		citecolor=blue
	}
	%Color links
	\usepackage{etoolbox}
	%Programming
	\usepackage[shortlabels]{enumitem}
	%Enable enumerate modification
	\usepackage{tcolorbox}
	%provides an environment for coloured and framed text boxes with a heading line.
	\tcbset{colback=white}
	%------------------------
	%Theorem-like Enviroments
	%------------------------
	\theoremstyle{plain}
	%This is the theorem style, plain means blod title and italic body
	\newtheorem{theorem}{Theorem}
	%This is how you define a theorem-like enviroment.
	\newtheorem{lemma}[equation]{Lemma}
	\newtheorem{proposition}[equation]{Proposition}
	%The option [equation] means the lemmas are numbered following the same system of equation.
	\newtheorem*{propstar}{Proposition}
	%Star version defines unnumbered enviroments
	\theoremstyle{definition}
	%This is the theorem style, definition means blod title and normal body
	\newtheorem{example}[equation]{Example}
	%The option [equation] means the examples are numbered following the same system of equation.
	\newtheorem{problem}{Problem}
	\crefname{problem}{problem}{problems}
	\Crefname{problem}{Problem}{Problems}
	%This is the enviroment used as problems in HWs.
	\theoremstyle{remark}
	%This is the theorem style, remark means italic title and normal body
	\newtheorem*{remark}{Remark}
	\newtheorem*{hint}{Hint}
	\newtheorem*{optional}{Optional}
	%Remark enviroment
	\numberwithin{equation}{problem}
	%Let equations be numbered with in each problems.
	\NewDocumentEnvironment{solution} {o+b} 
	{\begin{proof}[\IfNoValueTF{#1}{\textbf{Solution}}{\textbf{Solution} (#1)}]#2\end{proof}}{}
	%This defines the enviroment for you to write solutions.
	%------------------------
	\newlist{listinprob}{enumerate}{1}
	\setlist[listinprob]{label=(\alph{listinprobi}),
										ref=\theproblem.(\alph{listinprobi}),
										noitemsep}
	\crefname{listinprobi}{problem}{problems}
	\Crefname{listinprobi}{Problem}{Problems}
	%This defines a new list type used in the enviroment problem.
	%------------------------
	%DocumentCommands
	%------------------------
	%Notations for number sets
	\NewDocumentCommand \N {} { \mathbb{N} }%Natural numbers
	\NewDocumentCommand \Z {} { \mathbb{Z} }%Integers
	\NewDocumentCommand \Q {} { \mathbb{Q} }%Rational Numbers
	\NewDocumentCommand \R {} { \mathbb{R} }%Real Numbers
	\NewDocumentCommand \C {} { \mathbb{C} }%Complex Numbers
	\NewDocumentCommand \F {} { \mathbb{F} }%Field
	%Notations for maps
	\DeclareMathOperator{\id}{id} % identity
	\DeclareMathOperator{\pr}{pr} % projection
	\DeclareMathOperator{\pt}{pt} % point
	\DeclareMathOperator{\res}{res} % restriction
	%PairedDelimiters
	\DeclarePairedDelimiterX\abs[1]\lvert\rvert
		{ \ifblank{#1}{\:\cdot\:}{#1} }
	%abstract value function
	\DeclarePairedDelimiterX\norm[1]\lVert\rVert
		{ \ifblank{#1}{\:\cdot\:}{#1} }
	%the norm function
	\DeclarePairedDelimiterX\ceil[1]\lceil\rceil
		{ \ifblank{#1}{\:\cdot\:}{#1} }
	%ceil function
	\DeclarePairedDelimiterX\floor[1]\lfloor\rfloor
		{ \ifblank{#1}{\:\cdot\:}{#1} }
	%floor function
	\DeclarePairedDelimiterX\pairing[2]\langle\rangle
		{ \ifblank{#1}{\:\cdot\:}{#1}, \ifblank{#2}{\:\cdot\:}{#2} }
	\DeclarePairedDelimiterX\inner[2]\lparen\rparen
		{ \ifblank{#1}{\:\cdot\:}{#1}, \ifblank{#2}{\:\cdot\:}{#2} }
	%Inner product
	\providecommand\given{}
	\newcommand\SetSymbol[1][]
		{ \nonscript\:#1\vert\allowbreak\nonscript\:\mathopen{} }
	\DeclarePairedDelimiterX\Set[1]\{\}
		{ \renewcommand\given{\SetSymbol[\delimsize]}#1 }
	%a set
	%
	%Miscellaneous
	\NewDocumentCommand \vect { m } { \mathbf{#1} }
	%Vector
	\RenewDocumentCommand \le {} { \leqslant }
	\RenewDocumentCommand \ge {} { \geqslant }
	%Change the inequality symbols
\NewDocumentCommand \txforall {} {\quad\text{for all}\quad}
	%------------------------
	% Additional math symbols, used for the Math 110 course
% \DeclareMathOperator*\GCD{GCD}
\DeclareMathOperator*\lcm{lcm}
	%------------------------
	\allowdisplaybreaks
	%Allow a long equation to display in more than one page.

	%The following change the default layout of title
	\makeatletter
	\def\@maketitle{%
		\newpage
		\null
		\vskip 2em%
		\begin{center}%
			\let \footnote \thanks
			\sffamily 
			{\LARGE \@title \par}%
			\vskip 1.5em%
			{\large \@subtitle \par}%
			\vskip 1.5em%
			{\large Student name: \@author \par}%
			\ifdefempty{\@acknowledge}{}%
			{\large With help of: \@acknowledge \par}%
			\vskip 1em%
			{\large \@date}%
		\end{center}%
		\par
		\vskip 1.5em%
	}	
	\global\let\@subtitle\@empty
	\DeclareRobustCommand*{\subtitle}[1]{\gdef\@subtitle{#1}}
	\global\let\@acknowledge\@empty
	\DeclareRobustCommand*{\acknowledge}[1]{\gdef\@acknowledge{#1}}
	\makeatother
%Leave this change alone if you don't know what it means

%------------------------
%Information of the file
%------------------------
\title{Homework 4}
%The title of the file
\author{[Your Full Name Here]}
%Input your full name here.
% \acknowledge{[Your collaborators]}
%Input your collaborators, if none, comment it.
\subtitle{MATH 110~|~Introduction to Number Theory~|~Summer 2023}
%The subtitle
\date{\today} 
%The date, if commented, the value will be \today
 
%------------------------
%Document starts here
%------------------------
\begin{document}
\maketitle

\begin{quotation}
	Whenever you use a result or claim a statement, provide a \textbf{justification} or a \textbf{proof}, unless it has been covered in the class. In the later case, provide a \textbf{citation} (such as ``by the \emph{2-out-of-3 principle}'', ``by Coro. 0.31 in the textbook'', or ``by \cite[Coro. 0.31]{texbook}'').

	You are encouraged to \emph{discuss} the problems with your peers. However, you must write the homework \textbf{by yourself} using your words and \textbf{acknowledge your collaborators}.
\end{quotation}


\begin{problem}
  A \emph{Sophie Germain prime} is a prime number $p$ such that $2p + 1$ is also a prime. For example, $p = 2, 3, 5$ are Sophie Germain primes, but $p = 7$ is not (since $15 = 2\cdot 7 + 1$ is not a prime).\\
  \textbf{Prove that} if $p$ is a Sophie Germain prime, then $2p + 1$ is a divisor either of $2^p - 1$ or of $2^p + 1$, but not of both.
%----------------------------------------
\begin{solution} %Do not delete
WRITE YOUR SOLUTION HERE
\end{solution}\clearpage %Do not delete
%----------------------------------------
\end{problem}

\begin{problem}\label{Problem 14.2}
  \textbf{Find} the smallest positive integer $a$ such that $2^a \equiv 11 \pmod{23}$.
%----------------------------------------
\begin{solution} %Do not delete
WRITE YOUR SOLUTION HERE
\end{solution}\clearpage %Do not delete
%----------------------------------------
\end{problem}

\begin{problem}
	Let $p$ be a prime number. 
	\begin{listinprob}
		\item Let $f(T)$ be a polynomial modulo $p$ of degree $2$ or $3$. \textbf{Prove that} $f(T)$ is irreducible if and only if $f(T)$ has no roots modulo $p$.
		\begin{hint}
			Prove the contrapositive, looking at the degrees of the divisors of $f(T)$.
		\end{hint}
%----------------------------------------
\begin{solution} %Do not delete
WRITE YOUR SOLUTION HERE
\end{solution}\clearpage %Do not delete
%----------------------------------------
		\item \textbf{Count} the number of monic polynomials modulo $p$ of degree $d$.
%----------------------------------------
\begin{solution} %Do not delete
WRITE YOUR SOLUTION HERE
\end{solution}\clearpage %Do not delete
%----------------------------------------
		\item \textbf{Count} the number of monic irreducible polynomials modulo $p$ of degree $2$.
%----------------------------------------
\begin{solution} %Do not delete
WRITE YOUR SOLUTION HERE
\end{solution}\clearpage %Do not delete
%----------------------------------------
		\item \textbf{Count} the number of monic irreducible polynomials modulo $p$ of degree $3$.
%----------------------------------------
\begin{solution} %Do not delete
WRITE YOUR SOLUTION HERE
\end{solution}\clearpage %Do not delete
%----------------------------------------
	\end{listinprob}	
\end{problem}

\begin{problem}\label{Problem 15.3}
  For $n$ a nonzero integer, recall that $v_p(n)$ is the exponent of $p$ appearing in the prime factorization of $n$. Namely, $p^{v_p(n)}\mid n$, while $p^{v_p(n)+1}\nmid n$. Extend this definition to nonzero fractions as follows:
	\[
		v_p(\frac{n}{m}) := v_p(n) - v_p(m).
	\]
  \begin{listinprob}
		\item \textbf{Show that}, if the two fractions $\frac{n}{m}$ and $\frac{n'}{m'}$ represent the same rational number, then $v_p(\frac{n}{m})=v_p(\frac{n'}{m'})$.
%----------------------------------------
\begin{solution} %Do not delete
WRITE YOUR SOLUTION HERE
\end{solution}\clearpage %Do not delete
%----------------------------------------
  \end{listinprob}
	Hence, we obtain a function $v_p\colon\Q^{\times}\to\Z$. (Recall that $\Q^{\times}$ consists of nonzero rational numbers). The \textbf{$p$-adic norm} of a rational number $x$ is defined to be
	\[
		\abs*{x}_p:=
		\begin{dcases*}
			p^{-v_p(x)} & if $x\neq 0$;\\
			0 & if $x=0$.
		\end{dcases*}
	\]
	For example,
	\[
		\abs*{\frac{24}{25}}_{2} = \frac{1}{8},\qquad 
		\abs*{\frac{24}{25}}_{3} = \frac{1}{3},\qquad 
		\abs*{\frac{24}{25}}_{5} = 25.
	\]  
  \begin{listinprob}[resume]
		\item \textbf{Prove} the \emph{ultrametric triangle inequality}: for all $x,y\in\Q$,
		\[
			\abs*{x+y}_p \leq \max\Set*{\abs*{x}_p,\abs*{y}_p}.
		\]
%----------------------------------------
\begin{solution} %Do not delete
WRITE YOUR SOLUTION HERE
\end{solution}\clearpage %Do not delete
%----------------------------------------
		\item \textbf{Verify that}, the $p$-adic norm satisfies the three defining properties of a norm, namely:
		\begin{enumerate}%[\textbf{N\arabic*.}]
			\item $\abs*{x}_{p}=0$ if and only if $x=0$.
			\item $\abs*{xy}_{p}=\abs*{x}_{p}\abs*{y}_{p}$ for all $x,y\in\Q$.
			\item $\abs*{x+y}_{p}\le\abs*{x}_{p}+\abs*{y}_{p}$ for all $x,y\in\Q$.
		\end{enumerate}
%----------------------------------------
\begin{solution} %Do not delete
WRITE YOUR SOLUTION HERE
\end{solution}\clearpage %Do not delete
%----------------------------------------
		\item \label{p1.p-adic_to_congruence} \textbf{Show that}, for any two rational numbers $x$ and $y$, we have $\abs*{x-y}_{p}\le r$ if and only if $x\equiv y\pmod{p^e}$, where $e=\ceil{-\log_p(r)}$.
%----------------------------------------
\begin{solution} %Do not delete
WRITE YOUR SOLUTION HERE
\end{solution}\clearpage %Do not delete
%----------------------------------------
  \end{listinprob}
	Say a sequence $(x_n)_{n\in\N}$ of rational numbers is a \textbf{Cauchy sequence with respect to the $p$-adic norm} (a \textbf{Cauchy sequence} for short) if for every positive real number $\varepsilon>0$, there is a positive integer $N$ such that for all natural numbers $m,n>N$,  
	\[
		\abs*{x_m-x_n}_p<\varepsilon.
	\]
	Say a rational number $x\in\Q$ is the \textbf{limit} of a sequence $(x_n)_{n\in\N}$ of rational numbers \textbf{with respect to the $p$-adic norm} if for every positive real number $\varepsilon>0$, there is a positive integer $N$ such that for all natural numbers $n>N$,  
	\[
		\abs*{x_n-x}_p<\varepsilon.
	\]
	Say two Cauchy sequences $(x_n)_{n\in\N}$ and $(y_n)_{n\in\N}$ are \textbf{equivalent} if the sequence $(x_n-y_n)_{n\in\N}$ has the limit $0$.
	\begin{listinprob}[resume]
		\item \textbf{Prove that}, if a sequence $(x_n)_{n\in\N}$ of rational numbers has a limit $x\in\Q$ with respect to the $p$-adic norm, then it is a Cauchy sequence.
%----------------------------------------
\begin{solution} %Do not delete
WRITE YOUR SOLUTION HERE
\end{solution}\clearpage %Do not delete
%----------------------------------------
		\item Let $f(T)$ be an integer polynomial. \textbf{Show that}, if a sequence $(x_n)_{n\in\N}$ of rational numbers has a limit $x\in\Q$ with respect to the $p$-adic norm, then the sequence $(f(x_n))_{n\in\N}$ has the limit $f(x)$.
%----------------------------------------
\begin{solution} %Do not delete
WRITE YOUR SOLUTION HERE
\end{solution}\clearpage %Do not delete
%----------------------------------------
		\item \textbf{Deduce} the following version of \emph{Hensel's lifting} from the one in the lecture:
		\begin{quote}
			Let $f(T)$ be an integer polynomial. 
			If $x_0$ is an integer such that $\abs*{f(x_0)}<1$ but $\abs*{f'(x_0)}=1$, then it can be extended into a unique (up to equivalence) Cauchy sequence $(x_n)_{n\in\N}$ such that the sequence $(f(x_n))_{n\in\N}$ has the limit $0$ with respect to the $p$-adic norm. 
		\end{quote} 
		\begin{hint}
			Using \cref{p1.p-adic_to_congruence} to translate the statement in the language of congruence.
		\end{hint}
%----------------------------------------
\begin{solution} %Do not delete
WRITE YOUR SOLUTION HERE
\end{solution}\clearpage %Do not delete
%----------------------------------------
		\item However, a Cauchy sequence needs not to have a limit in $\Q$ with respect to the $p$-adic norm. 
		\textbf{Give such a counterexample}.
		\begin{remark}
			Lack of limits in $\Q$ is one motivation to introduce \emph{$p$-adic numbers}.
		\end{remark}
%----------------------------------------
\begin{solution} %Do not delete
WRITE YOUR SOLUTION HERE
\end{solution}\clearpage %Do not delete
%----------------------------------------
	\end{listinprob}
\end{problem}



%%%	List your reference here
%------------------------
\begin{thebibliography}{9}  %Do not delete
%List your references here
\bibitem[TEXT]{texbook}
\emph{An Illustrated Theory of Numbers}, Martin H. Weissman.

\bibitem[label]{citekey}
[Book(s): \emph{Title}, Author ] or [Online: \href{http://example.com/}{Link}]
\end{thebibliography}  %Do not delete
%------------------------
\end{document}
