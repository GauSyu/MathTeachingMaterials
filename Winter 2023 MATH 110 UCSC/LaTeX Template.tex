%LaTeX Template 
%%	by	Xu Gao (gausyu@gmail.com)
%%LPPL: http://www.latex-project.org/lppl.txt
%------------------------
%The following is the Preamble
%------------------------
\documentclass[11pt]{article}
%This is the document class. 
%%	Most common: article, beamer, book, etc.
%%	11pt means the default font size is 11pt. One can also use 10pt, 11pt, or 12pt.
%%	See https://en.wikibooks.org/wiki/LaTeX/Document_Structure#Document_classes for explaination.
%------------------------
%Packages
%------------------------
\usepackage[a4paper,total={6in, 9in}]{geometry} 
%This aims to customize page layout
\usepackage[T1]{fontenc}
%Using T1 font
\usepackage{mathtools}
%The main MATH package (for whom may wonder, mathtools load amsmath automatically, so no need \usepackage{amsmath})
\usepackage{amsthm}
%This defines the theorem enviroments
\usepackage{amssymb}
%Math symbols
\usepackage{bm}
%Bold math fonts, provide \bm{} command
\usepackage[scr=rsfs,cal=euler]{mathalpha}
%Math fonts 
%%The package provides means of loading maths alphabets (such as are normally addressed via macros \mathcal, \mathbb, \mathfrak and \mathscr)
%%How to use? "scr=" set what font \mathscr use, "cal=" set what font \mathcal use, "bb=" set what font \mathbb use, and "frak=" set what font \mathfrak use.
\usepackage{tikz}
%Drawing
\usepackage{graphicx}
%Support for graphics
\usepackage{hyperref}
\usepackage[nameinlink]{cleveref}
%Hyper links
\hypersetup{
	colorlinks=true,
	linkcolor=blue,
	urlcolor=magenta
}
%Color links
\usepackage{etoolbox}
%Programming
\usepackage{enumerate}
%Enable enumerate modification
%------------------------
%Theorem-like Enviroments
%------------------------
\theoremstyle{plain}
%This is the theorem style, plain means blod title and italic body
\newtheorem{theorem}{Theorem}
%This is how you define a theorem-like enviroment.
\newtheorem{lemma}[equation]{Lemma}
%The option [equation] means the lemmas are numbered following the same system of equation.
\theoremstyle{definition}
%This is the theorem style, definition means blod title and normal body
\newtheorem{example}[equation]{Example}
%The option [equation] means the examples are numbered following the same system of equation.
\newtheorem{problem}{Problem}
%This is the enviroment used as problems in HWs.
\theoremstyle{remark}
%This is the theorem style, remark means italic title and normal body
\newtheorem*{remark}{Remark}
%Remark enviroment
\numberwithin{equation}{problem}
%Let equations be numbered with in each problems.
\newenvironment{solution}{\begin{proof}[\textbf{Solution}]}{\end{proof}}
%This defines the enviroment for you to write solutions.
%------------------------
%DocumentCommands
%------------------------
%Notations for number sets
\NewDocumentCommand \N {} { \mathbb{N} }%Natural numbers
\NewDocumentCommand \Z {} { \mathbb{Z} }%Integers
\NewDocumentCommand \Q {} { \mathbb{Q} }%Rational Numbers
\NewDocumentCommand \R {} { \mathbb{R} }%Real Numbers
\NewDocumentCommand \C {} { \mathbb{C} }%Complex Numbers
\NewDocumentCommand \F {} { \mathbb{F} }%Field
%Notations for maps
\DeclareMathOperator{\id}{id} % identity
\DeclareMathOperator{\pr}{pr} % projection
\DeclareMathOperator{\pt}{pt} % point
\DeclareMathOperator{\res}{res} % restriction
%PairedDelimiters
\DeclarePairedDelimiterX\abs[1]\lvert\rvert
  { \ifblank{#1}{\:\cdot\:}{#1} }
%abstract value function
\DeclarePairedDelimiterX\norm[1]\lVert\rVert
  { \ifblank{#1}{\:\cdot\:}{#1} }
%the norm function
\DeclarePairedDelimiterX\ceil[1]\lceil\rceil
  { \ifblank{#1}{\:\cdot\:}{#1} }
%ceil function
\DeclarePairedDelimiterX\floor[1]\lfloor\rfloor
  { \ifblank{#1}{\:\cdot\:}{#1} }
%floor function
\DeclarePairedDelimiterX\pairing[2]\langle\rangle
  { \ifblank{#1}{\:\cdot\:}{#1}, \ifblank{#2}{\:\cdot\:}{#2} }
\DeclarePairedDelimiterX\inner[2]\lparen\rparen
  { \ifblank{#1}{\:\cdot\:}{#1}, \ifblank{#2}{\:\cdot\:}{#2} }
%Inner product
\providecommand\given{}
\newcommand\SetSymbol[1][]
  { \nonscript\:#1\vert\allowbreak\nonscript\:\mathopen{} }
\DeclarePairedDelimiterX\Set[1]\{\}
  { \renewcommand\given{\SetSymbol[\delimsize]}#1 }
%a set
%
%Miscellaneous
\NewDocumentCommand \vect { m } { \mathbf{#1} }
%Vector
%------------------------

\allowdisplaybreaks
%Allow a long equation to display in more than one page.

%The following change the default layout of title
\makeatletter
\def\@maketitle{%
	\newpage
	\null
	\vskip 2em%
	\begin{center}%
		\let \footnote \thanks
		\sffamily 
		{\LARGE \@title \par}%
		\vskip 1.5em%
		{\large
		\lineskip .5em%
		\begin{tabular}[t]{c}%
		\@author \\[1em]
		\@subtitle
		\end{tabular}\par}%
		\vskip 1em%
		{\large \@date}%
	\end{center}%
	\par
	\vskip 1.5em%
}	
\global\let\@subtitle\@empty
\DeclareRobustCommand*{\subtitle}[1]{\gdef\@subtitle{#1}}
\makeatother
%Leave this change alone if you don't know what it means

%------------------------
%Information of the file
%------------------------
\title{A {\LaTeX} template}
%The title of the file
\author{Xu Gao}
%The author
\subtitle{MATH 110~|~Introduction to Number Theory~|~Winter 2023}
%The subtitle
% \date{} 
%The date, if commented, the value will be \today
 
%------------------------
%Document starts here
%------------------------
\begin{document}
\maketitle

\section{Basic use}
Just type in words. The {\LaTeX} will handle the typesetting. 
The words will use the default font. 
The command \verb|\emph{text}| will emphasize the word \emph{text}. 
In usual context, it behaves as \verb|\textit|, but see how does it behave in an italic context: \textit{this is in italic, while a \emph{text} is emphasized}. 
To obtain \textbf{bold} words, use \verb|\textbf|. 

Skipping one or more lines will start a new paragraph with indentation. 
If you do not want the indentation, put \verb|\noindent| at the beginning of the paragraph. 
If you start a new line without skip, then you are still in the same paragraph. \\
The command \verb|\\| will start a new line without leave the current paragraph. 

See \href{https://www.overleaf.com/learn/latex/Articles/How_to_change_paragraph_spacing_in_LaTeX}{This article} for how to change paragraph spacing. Note that how I create a hyperlink to a URL. It has a color since I used \verb|colorlinks=true| in \verb|\hypersetup|.

You can use defined commands such as \verb|\LaTeX| to input some symbols. 
However, math symbols should be putted in math environments. 
The basic math environment is the \emph{in-line math mode}: \verb|$...$|. There are also \emph{display math modes}: \verb|\[...\]|, for example:
\[
	5+7=12.
\] 

\section{Math symbols}\label{sec:2}
There are many predefined some math symbols and I also have defined some. 
The following is a non-complete list of math symbols which will be used in this course. 
\begin{itemize}
	\item Subsets: $A\subset B$, $A\subseteq B$, and $A\subsetneq B$. 
	\item Union $A\cup B$, intersection $A\cap B$, set minus $A\setminus B$, and quotient set $A/B$. 
	\item Use \verb|\Set| to create a set: $\Set*{ \text{elements} }$ or $\Set*{ \text{elements} \given \text{conditions} }$. The star \verb|*| in \verb|\Set*| means that the brackets will be scaled automatically to match the size of its context. See the followings:
	\[
		\Set*{	(x,y,z)\in\Z^3	\given	x^p+y^p=z^p	},
		\qquad
		\Set*{	\frac{a+b\sqrt{5}}{2}	\given	a,b\in\Z	}.
	\]
	Note that how \verb|\text| allows we to input text in math mode. Be aware that there is no space between math symbols and the contents of \verb|\text|: see $t\text{h}is$. 
	\item Fractions: \verb|\tfrac| (in-line fraction), \verb|\frac| (display fraction), and \verb|\cfrac| (continued fraction). 
	\[
		\tfrac{1}{a_1+\tfrac{1}{a_2+\cdots}},
		\frac{1}{a_1+\frac{1}{a_2+\cdots}},
		\cfrac{1}{a_1+\cfrac{1}{a_2+\cdots}}.
	\]
	\item The set of Natural numbers $\N$, the set of integers $\Z$, the set of rational numbers $\Q$, the set of real numbers $\R$, and the set of complex numbers $\C$. 
	\item The identity map $\id$, the projection map $\pr$, and the restriction map $\res$.
	\item The abstract value $\abs{}$, the norm $\norm{}$, the ceil $\ceil{}$, and the floor $\floor{}$. They have star-variant and different size variants. See the followings: 
	\[
		\abs{\frac{a}{b}},
		\abs*{\frac{a}{b}},
		\abs[\big]{\frac{a}{b}},
		\abs[\Big]{\frac{a}{b}},
		\abs[\bigg]{\frac{a}{b}},
		\abs[\Bigg]{\frac{a}{b}},
	\]
	\item Sum \verb|\sum| and product \verb|\prod|. $\sum_{i=1}^{n}a_n$, $\prod_{i=1}^{n}a_n$, 
	\[
		\sum_{i=1}^{n}a_n,\qquad
		\prod_{i=1}^{n}a_n
	\] 
	Compare the in-line ones and the display ones. \\
	Aside: \verb|\quad| and \verb|\qquad| display some spaces, which is useful since the math mode ignores the spaces. 
	\item Inner product $\inner{a}{b}$.
	\item Vectors $\vect{u}$, $\vect{v}$, $\vect{x}$.
	\item Math fonts: \verb|\mathcal| $\mathcal{A}$, \verb|\mathscr| $\mathscr{A}$, \verb|\mathfrak| $\mathfrak{A}$, \verb|\mathbb| $\mathbb{A}$, and \verb|\mathbf| $\mathbf{A}$.
\end{itemize}

\section{Equations}\label{sec:3}
There are many equation environments for different purposes.
\begin{enumerate}
	\item The \verb|equation| provides an equation with numbers. 
	\begin{equation}\label{Alabel}
		A+B=C.
	\end{equation}
	Here I \verb|\label| this equation, so we can refer to it using \verb|\cref|. 
	See: \cref{Alabel}.
	\item The star-variant \verb|equation*| is the same as \verb|\[...\]|.
	\item One can use \verb|split| in side an equation to input aligned multiline equations. 
	\begin{equation}
		\begin{split}
			A &= \frac{\pi r^2}{2} \\
			&= \tfrac{1}{2}\pi r^2.
		\end{split}
	\end{equation}
	You can put more than ones in the same \verb|equation|:
	\begin{equation}
		\begin{split}
			A &= \frac{\pi r^2}{2} \\
			&= \tfrac{1}{2}\pi r^2.
		\end{split}
		\qquad
		\text{and}
		\qquad
		\begin{split}
			V &= \frac{4\pi r^3}{3} \\
			&= \tfrac{4}{3}\pi r^3.
		\end{split}
	\end{equation}
	\item The \verb|align| environment: multi equations with alignments.
	\begin{align}
		\framebox[1cm][c]{} &= \framebox[4cm][c]{} \\
		\framebox[2cm][c]{} &= \framebox[3.5cm][c]{}
	\end{align}
	It also has a star-variant which has no numbers.
	\item There is a \verb|\MoveEqLeft| command move the equation in this line slightly left (can be specified with \verb|[number]|). 
	\begin{align*}
		\MoveEqLeft \framebox[10cm][c]{Long first line}\\
		& = \framebox[6cm][c]{2nd line}\\
		& \vdotswithin{=}\\
		& = \framebox[6cm][c]{last line}
	\end{align*}
	(see the code for more details such as the use of \verb|\vdotswithin{=}|)
	\item There is also many \verb|cases| environments:
	\[
		f(x) =
		\begin{cases}
			\sum_{i=1}^{n}a_i(x) &= \text{condition 1}, \\
			\frac{1}{x} &= \text{condition 2}.
		\end{cases}\qquad
		f(x) =
		\begin{cases*}
			\sum_{i=1}^{n}a_i(x) &= condition 1, \\
			\frac{1}{x} &= condition 2.
		\end{cases*}
	\]
	The \verb|cases| and \verb|cases*| provide in-line formulas, while the following \verb|dcases| and \verb|dcases*| provide display mode:
	\[
		f(x) =
		\begin{dcases}
			\sum_{i=1}^{n}a_i(x) &= \text{condition 1}, \\
			\frac{1}{x} &= \text{condition 2}.
		\end{dcases}\qquad
		f(x) =
		\begin{dcases*}
			\sum_{i=1}^{n}a_i(x) &= condition 1, \\
			\frac{1}{x} &= condition 2.
		\end{dcases*}
	\]
\end{enumerate}

\section{Lists}
There are three lists: \verb|itemize|, \verb|enumerate|, and \verb|description|
\begin{description}
	\item[itemize] the list in \cref{sec:2} is such a one;
	\item[enumerate] the list in \cref{sec:3} is such a one;
	\item[description] this list is such a one.
	\item[Tag] indeed, one can change any the tag of any item in any kind of list as what I do in this list.
\end{description}
One can also change the numbering of a \verb|enumerate| list as follows:
\begin{enumerate}[(a).]
	\item abaaba
	\item balbla
\end{enumerate}

\section{Environments}
There are many theorem-like environments. 
We will mainly use \verb|problem| and \verb|solution|. 
Here is an example
\begin{problem}\label{p:1}
	This is a problem. 
\end{problem}
\begin{solution}
	This should be your solution.

	For your convenience, I have already let equations and lemmas be numbered within problems. See this 
	\begin{equation}
		\framebox{This is an equation}
	\end{equation} 
	and this 
	\begin{lemma}
		A lemma used to solve \Cref{p:1}.
	\end{lemma}

	When the \verb|solution| environment ends, there will be a QED mark:
\end{solution}

\section{Compile}
You can use pdf\LaTeX{} as the compiler, which is also the default one for many online editors. Note that to obtain correct cross-references, you may need to compile the source twice.

\vskip 3em
\textsf{\Large\color{teal} READ the .tex file to see how this document is made and start your {\LaTeX} journey by playing with this one.}\\
You will learn more (on both math symbols and typesetting) as the course proceeding.
\end{document}
